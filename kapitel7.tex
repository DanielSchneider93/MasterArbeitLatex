\chapter{Zusammenfassung}


Fortschritte in der Photogrammetrie sind viel zu sehr mit den Fortschritten der Computer Vision verflochten, als dass sich die Konvergenz der beiden Disziplinen umkehren wird. Photogrammetrie und Computer Vision haben die Extraktion und Rekonstruktion von Daten aus Bildmaterial zu unterschiedlichen Zeiten und mit unterschiedlichen Zielen begonnen. Als sich dann 3D-Modelle als Referenzziel herausstellten, wurde der Austausch von Ansätzen und Techniken zwischen den Disziplinen vorangetrieben (vgl. \cite{state_of_art} S.9). Dies hat dazu geführt, dass Photogrammetrie und Computer Vision verfahrenstechnisch kaum mehr unterscheidbar sind, auch wenn die ursprünglichen Anwendungsgebiete verschieden sind. 

Im Rahmen dieser Arbeit wurden Algorithmen und Konzepte aus der Photogrammetrie und der Computer Vision beschrieben und verglichen. Anschließend wurde ein Überblick über die Ziele, Unterschiede und Gemeinsamkeiten von Photogrammetrie, SLAM und SfM gegeben. Es hat sich gezeigt, dass sich diese Verfahren inhaltlich stark überschneiden, auch wenn die Fachrichtung, der historische Kontext und die Ziele von Photogrammetrie und SLAM grundverschieden sind. 

Im Praxisteil wurde eine Applikation erstellt, welche ARCore verwendet. ARCore implementiert SLAM und verwendet damit auch photogrammetrische Verfahren. Photogrammetrie kann also in Augmented Reality Anwendungen verwendet werden, auch wenn dann die Bezeichnung irreführend ist. SLAM könnte man als Echtzeit-Photogrammetrie bezeichnen, welche nicht auf größtmögliche Genauigkeit, sondern auf Schnelligkeit und Stabilität ausgelegt ist. In Zukunft wird sich die Konvergenz der in dieser Arbeit beschrieben Disziplinen noch verstärken, die Kategorisierung in die verschiedenen Fachbereiche ist jedoch, mit Sicht auf die Anwendungsgebiete, Philosophien und den historischen Ursprung, sinnvoll.

Vielen Dank an Prof. Dr. Tobias Lenz und Prof. Dr. Klaus Jung die Betreuung und Korrektur dieser Arbeit.
	
