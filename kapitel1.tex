\chapter{Einführung und Motivation}

Photogrammetrie ist "die Wissenschaft und Technologie der Gewinnung von Informationen
über die physische Umwelt aus Bildern, mit einem Schwerpunkt auf Vermessung,
Kartierung und hochgenauer Messtechnik". (Heipke, 2017, S.5 \cite{photo})  Die Photogrammetrie beschäftigt sich  mit der Rekonstruktion von dreidimensionalen Daten aus zweidimensionalen Informationsträgern, wie Bildern oder Laserscan Daten. Dabei gehen diese Daten alle auf das Prinzip der Aufnahme der elektromagnetischen Strahlung zurück. Bei Bildern ist das die Helligkeits und Farbverteilung, bei Laserscans, Entfernungsbilder, beziehungsweise Punktwolken. Die Disziplin der Photogrammetrie ist dabei dem Bereich der Fernerkundung zuzuordnen, die sich mit der Auswertung von geometrischen oder semantischen Informationen beschäftigt. Beides sind Fachbereiche, die sich über die Jahrzehnte entwickelt haben und sich dem Gebiet der Geodäsie zuordnen lassen. Die Geodäsie erfasst Geoinformationen über die Erde, die dann beispielsweise mit Kartographie visualisiert werden können. Das Gebiet der Computer Vision, das sich großteils parallel mit der Photogrammetrie entwickelt hat, verfolgt den gleichen Ansatz. Auch hier ist die Auswertung digitaler Bilder das zentrale Element. Nachdem sich Photogrammetrie und Computer Vision lange unabhängig voneinander entwickelt haben, ist heute Photogrammetrie als Grundlage von Computer Vision anerkannt. (vgl. \cite{photo} S. 5-7)

Durch die rasante Entwicklung der Technik und der damit eingehenden Steigerung der Rechenpower, sind die Bereiche, in denen Photogrammetrie eingesetzt werden kann gestiegen. Im Rahmen dieser Arbeit soll evaluiert werden, ob photogrammetrische Verfahren bei Smarthphones eingesetzt werden können, um in Echtzeit aus Videodaten die dreidimensionalen Beschaffenheit der gefilmten Objekte zu rekonstruieren. Weiterhin soll dieses Verfahren genutzt werden um anschließend ein dreidimensionales Koordinatensystem mit Tiefeninformationen zu erzeugen, in welches dann dynamische Standortinformationen, im Stil von Augumented Reality, eingeblendet werden können. Dazu ist im Rahmen dieser Arbeit eine auf dem Android Betriebsystem basierende Anwendung erstellt worden.



  


