\chapter{Motivation}

Photogrammetrie ist \glqq die Wissenschaft und Technologie der Gewinnung von Informationen
über die physische Umwelt aus Bildern, mit einem Schwerpunkt auf Vermessung,
Kartierung und hochgenauer Messtechnik\grqq{}. (Heipke, 2017, S.5 \cite{photo})  Die Photogrammetrie beschäftigt sich  mit der Rekonstruktion von dreidimensionalen Daten aus zweidimensionalen Informationsträgern, wie Bildern oder Laserscan Daten. Dabei gehen diese Daten alle auf das Prinzip der Aufnahme der elektromagnetischen Strahlung zurück. Bei Bildern ist das die Helligkeits und Farbverteilung, bei Laserscans sind es Entfernungsbilder, beziehungsweise Punktwolken. Die Disziplin der Photogrammetrie ist dabei dem Bereich der Fernerkundung zuzuordnen, die sich mit der Auswertung von geometrischen oder semantischen Informationen beschäftigt. Beides sind Fachbereiche, die sich über die Jahrzehnte entwickelt haben und sich dem Gebiet der Geodäsie zuordnen lassen. Die Geodäsie erfasst Geoinformationen über die Erde, die dann beispielsweise mit Kartographie visualisiert werden können. Das große Potenzial der Bereitstellung und Gewinnung von dreidimensionalen Daten war in der Photogrammetrie seit Beginn eine der wichtigsten Stärken der Technologie. Photogrammetrie wird von Raumsonden im All bis hin zur Mikroskopie eingesetzt. Auch wenn die ursprüngliche Kernanwendung die Kartenerstellung war, sind nicht-topographische Anwendungen seit der Einführung digitaler Kameras stark expandiert. Dazu zählt zum Beispiel die hochpräzise industrielle Messtechnik, die architektonische Photogrammetrie, die medizinische und forensische Bildgebung und Analyse, sowie die Erstellung von dreidimensionalen Modellen aus Fotoserien oder Laserdaten, welche der Nahbereichsphotogrammetrie zugeordnet werden kann (vgl. \cite{state_of_art} S.1).


Das Gebiet der Computer Vision (CV), das sich in den letzten Jahren teilweise parallel mit der Photogrammetrie entwickelt hat, verfolgt einen ähnlichen Ansatz und ist deshalb inhaltlich mit der digitalen Photogrammetrie verwandt. In diesem Bereich haben sich ebenfalls verschiedene Verfahren zur Kartierung der Umwelt und der gleichzeitigen Lokalisierung in dieser entwickelt, welche ebenfalls in dieser Arbeit beschrieben werden.  \\ \\

Das Forschungsziel der Computer Vision ist es, anhand von zweidimensionalen Bilder oder Videos die geometrischen Informationen von dreidimensionalen Objekten, wie Form, Position, räumlicher Lage, Bewegung oder Daten über den Inhalt der Bilder zu gewinnen, sowie dem Computer ein umfassenes \glqq Verständnis\grqq{} der Inhalte der Bilder zu ermöglichen (vgl. \cite{photo} S. 5-7).

Durch die Entwicklung der Technik der letzten Jahrzehnte und der damit eingehenden Steigerung der Rechenpower bei mobilen Endgeräten, haben sich die Bereiche vergrößert, in denen digitale Photogrammetrie, sowie Verfahren aus der CV eingesetzt werden können. Im Rahmen dieser Arbeit soll evaluiert werden, ob photogrammetrische Verfahren für Augmented Reality Anwendungen im Bereich von Smarthphones eingesetzt werden können, um in Echtzeit aus Videodaten die dreidimensionale Beschaffenheit der gefilmten Objekte zu rekonstruieren. Dazu werden, nach einer kurzen Definition und Einführung in Augmented Reality, die Verfahren und Algorithmen aus der Photogrammetrie und der Computer Vision analysiert und verglichen. Es werden weiterhin Vorteile sowie Nachteile der jeweiligen Disziplinen aufgezeigt und anschließend im Kontext zu Augmented Reality beschrieben. 

Im Rahmen dieser Arbeit ist weiterhin eine auf dem Android Betriebsystem basierende Anwendung erstellt worden, welche eine der neuen Technologien im Bereich Augmented Reality implementiert und aktuelle Frameworks und Bibliotheken verwendet. Die Besonderheiten dieser Frameworks werden im Praxisteil dieser Arbeit beschrieben.



  


