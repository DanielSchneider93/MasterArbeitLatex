\chapter{Motivation}

Photogrammetrie ist "die Wissenschaft und Technologie der Gewinnung von Informationen
über die physische Umwelt aus Bildern, mit einem Schwerpunkt auf Vermessung,
Kartierung und hochgenauer Messtechnik". (Heipke, 2017, S.5 \cite{photo})  Die Photogrammetrie beschäftigt sich  mit der Rekonstruktion von dreidimensionalen Daten aus zweidimensionalen Informationsträgern, wie Bildern oder Laserscan Daten. Dabei gehen diese Daten alle auf das Prinzip der Aufnahme der elektromagnetischen Strahlung zurück. Bei Bildern ist das die Helligkeits und Farbverteilung, bei Laserscans sind es Entfernungsbilder, beziehungsweise Punktwolken. Die Disziplin der Photogrammetrie ist dabei dem Bereich der Fernerkundung zuzuordnen, die sich mit der Auswertung von geometrischen oder semantischen Informationen beschäftigt. Beides sind Fachbereiche, die sich über die Jahrzehnte entwickelt haben und sich dem Gebiet der Geodäsie zuordnen lassen. Die Geodäsie erfasst Geoinformationen über die Erde, die dann beispielsweise mit Kartographie visualisiert werden können. Das große Potenzial der Bereitstellung und Gewinnung von dreidimensionalen Daten, war in der Photogrammetrie seit Beginn eine der wichtigsten Stärken der Technologie. Photogrammetrie wird von Raumsonden im All bis hin zur Mikroskopie eingesetzt. Auch wenn die Kartenerstellung die ursprüngliche Kernanwendung war, sind nicht-topographische Anwendungen, seit der Einführung digitaler Kameras stark expandiert. Dazu zählt zum Beispiel die hochpräzise industrielle Messtechnik, die architektonische Photogrammetrie, die medizinische und forensische Bildgebung und Analyse sowie eine Reihe weiterer Verfahren. (vgl. \cite{state_of_art} S.1)


Das Gebiet der Computer Vision, das sich großteils parallel mit der Photogrammetrie entwickelt hat, verfolgt den gleichen Ansatz und ist inhaltlich sehr stark mit der digitalen Photogrammetrie verbunden. Das Forschungsziel der Computer Vision ist es, anhand von zeidimensionalen Bilddaten die geometrischen Informationen von dreidimensionalen Objekten, wie Form, Position, räumlicher Lage, Bewegung oder ähnliche Daten zu gewinnen. Unter diesem Gesichtspunkt gibt es sehr viele Gemeinsamkeiten zwischen Computer Vision und digitaler Photogrammetrie. (vgl. \cite{pose_est_epi} S.1) Nachdem sich Photogrammetrie und Computer Vision lange unabhängig voneinander entwickelt haben, ist Photogrammetrie heute als Grundlage von Computer Vision anerkannt. (vgl. \cite{photo} S. 5-7)

Durch die Entwicklung der Technik und der damit eingehenden Steigerung der Rechenpower im mobilen Bereich, haben sich die Bereiche, in denen Photogrammetrie eingesetzt werden kann vergrößert. Im Rahmen dieser Arbeit soll evaluiert werden, ob photogrammetrische Verfahren für Augmented Reality Anwendungen im Bereich von Smarthphones eingesetzt werden können, um in Echtzeit aus Videodaten die dreidimensionalen Beschaffenheit der gefilmten Objekte zu rekonstruieren. Dazu wird die photogrammetrische Pipeline analysiert und mit aktuellen Verfahren verglichen.

Im Rahmen dieser Arbeit ist ebenfalls eine auf Android basierende Anwendung erstellt worden, welche eine der neuen Technologien im Bereich Augmented Reality implementiert.



  


