\chapter{Vergleich von Photogrammetrie und SLAM}

\section{Ähnlichkeiten und Unterschiede}
Es ist inzwischen allgemein bekannt, dass Photogrammetrie und geometrische Computer Vision zwei eng zusammenhängende Disziplinen sind. Sie haben viele ähnliche Aufgabenstellungen und Ziele, wie Kalibrierung, Orientierung und Rekonstruktion. Viele Arbeiten und Forschungen beziehen sich auf beide Gebiete, wie relative Orientierung (Philip, 1996; Nistèr, 2004), die räumliche Analyse von Einzelbildern (Masry, 1981; Lepetit et al., 2009), Feature Erkennung  (Förstner \& Gülch, 1986;
Lowe, 2004) oder etwa der Bündelblockausgeleich  (Triggs et al., 2000). Dabei sollte beachtet werden, dass viele dieser Probleme erst in der Photogrammetrie untersucht und beschrieben worden sind und erst später in der Computer Vision signifikant weiter entwickelt wurden. Dies hat die Kommunikation zwischen beiden Fachbereichen gefördert. (vgl. \cite{ph_vs_cv} S.93)

\url{https://elib.uni-stuttgart.de/bitstream/11682/3934/1/Tang_Uni.pdf} 

Seite 93, Unterschiede und gemeinsamkeiten 

\section{Analyse der Echtzeitfähigkeit}


\url{https://phowo.ifp.uni-stuttgart.de/publications/phowo05/280foerstner.pdf}


