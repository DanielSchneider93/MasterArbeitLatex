\documentclass[12pt,oneside]{scrreprt}
\usepackage[T1]{fontenc}		% Einstellungen fuer Umlaute usw.
\usepackage[utf8x]{inputenc}
\usepackage[ngerman]{babel}
\usepackage{listings}
\usepackage{float}
\usepackage{fancyvrb}
\usepackage{parskip}
\usepackage{setspace}
\usepackage[official]{eurosym}			% Einstellungen fuer Absaetze: Abstand statt Einrueckung
\addtokomafont{disposition}{\rmfamily}

\usepackage[a4paper,			% Papierformat A4
	    left=2.5cm,				% linker Rand
	    right=2.5cm,			% rechter Rand
	    top=1.5cm,				% oberer Rand 
	    bottom=1.5cm,			% unter Rand
	    marginparsep=5mm,		% Abstand der Randnotizen
	    marginparwidth=10mm, 	% Breite der Randnotizen
	    headheight=7mm,			% Hoehe der Kopfzeile
	    headsep=1.2cm,			% Abstand der Kopfzeile
	    footskip=1.5cm,			% Abstand der Fusszeile
	    includeheadfoot]{geometry}

\usepackage{fancyhdr}						% Konfiguration von Kopf- und Fusszeilen
\pagestyle{fancy}							% Seitenstil 'fancy'
\fancyhf{}									% vorhandene Einstellungen loeschen
\setlength{\headwidth}{\textwidth}			% Kopf- und Fusszeile so breit wie der Haupttext
\fancyfoot[R]{\thepage} 					% Festlegung des Seitenstils: Seitenzahlen in der Fusszeile rechts
\fancyfoot[L]{\leftmark}					% Kapitelnr. und -Bezeichnung in der Fusszeile links
\fancyhead[R]{\IhreArbeit}					% "Bachelorarbeit" in der Kopfzeile rechts
\fancyhead[L]{\IhrVorname\ \IhrNachname}	% Vorname und Name in der Kopfzeile links
\renewcommand{\chaptermark}[1]{			% Definition der Ausgabe des Kapitels
  \markboth{Kapitel \thechapter. #1}{}}
\renewcommand{\headrulewidth}{0.5pt}		% Trennlinie zwischen Kopfzeile und Haupttext
\renewcommand{\footrulewidth}{0.5pt}		% Trennlinie zwischen Haupttext und Fusszeile
\fancypagestyle{plain}{					% Anpassung des Seitenstils 'plain' bei Beginn neuer Kapitel
  \fancyhf{}								% Vorbelegung loeschen
  \fancyfoot[C]{\thepage}					% Seitenzeilen in der Fusszeile mittig
  \fancyhead[R]{\IhreArbeit}				% "Bachelorarbeit" in der Kopfzeile rechts
  \fancyhead[L]{\IhrVorname\ \IhrNachname}	% Vorname und Name in der Kopfzeile links
}

\usepackage{amsmath}			% Pakete fuer den Mathematikmodus
\usepackage{amssymb}
\usepackage[intlimits]{empheq}

\usepackage[sc]{mathpazo}		% Schriftart Palatino fuer Haupttext und Mathematikmodus
\usepackage{pifont}				% zusaetzliche Symbole
\usepackage{csquotes}

\usepackage[format=hang,		% Einstellung fuer Bildunterschriften
            font={footnotesize},
            labelfont={bf},
            margin=1cm,
            aboveskip=5pt,
            position=bottom]{caption}

\usepackage{graphicx}							% Einbinden von Graphiken
\usepackage[svgnames,table,hyperref]{xcolor} 	% Verwendung von Farben
\usepackage{tikz}								% Erstellen von Grafiken
\usetikzlibrary{positioning,arrows,plotmarks} % TikZ-Bibliotheken
%\usepackage{pgfplots}                           % Darstellung von Plots, Funktionen, Graphen usw.

%
% Weitere Pakete
%
%\usepackage{listings}			% Darstellung von Quellcode
%\lstset{language=Python, basicstyle=\ttfamily, numbers=none}
%
%\usepackage[european, siunitx]{circuitikz}	% Darstellung von Schaltungen
%
%\usepackage{enumerate}			% Formatierung nummerierter Listen

\usepackage{microtype,relsize}					% Wird verwendet, um Nachnamen auf Titelseite gesperrt darzustellen
\newcommand*{\Sperren}[1]{\textls*[100]{#1}}

% 
% Persoenliche Angaben
% 
% 
\newcommand*{\IhrVorname}{Daniel}
\newcommand*{\IhrNachname}{Schneider}
\newcommand*{\IhrStudiengang}{Internationaler Studiengang Medieninformatik}
\newcommand*{\IhreArbeit}{Masterarbeit}
\newcommand*{\IhrTitelDE}{Photogrammetrie zur Platzierung von standortbezogenen dynamischen Inhalten in AR}
\newcommand*{\IhrTitelEN}{Photogrammetry for placement of location-based dynamic content in AR}
\newcommand*{\IhrBearbeitungszeitraumVON}{13.05.2019}
\newcommand*{\IhrBearbeitungszeitraumBIS}{16.09.2019}
\newcommand*{\IhrErstpruefer}{Prof. Dr. Tobias Lenz}
\newcommand*{\IhrZweitpruefer}{Prof. Dr. Klaus Jung}
\newcommand*{\IhreFirma}{HTW Berlin}
\newcommand*{\IhrFirmenbetreuer}{}
\newcommand*{\IhreZusammenfassung}{ Im Rahmen dieser Arbeit werden die wichtigsten Verfahren aus der Photogrammetrie und der Computer Vision, wie der Bündelblockausgleich, SLAM (Simultaneous Location and Mapping), SfM (Structure from Motion) im Kontext zu Augmented Reality (AR) für Android Smartphones analysiert. Weiterhin wird versucht die Algorithmen der beiden Disziplinen in Kontext zu stellen und eine Unterscheidung, trotz der inhaltlichen Überschneidung der Algorithmen, zu formulieren. Die Tauglichkeit der photogrammetrischen Verfahren wird in Bezug zu Augmented Reality geprüft. Im Praxisteil dieser Arbeit wird eine Anwendung erstellt, die aktuelle AR Frameworks verwendet, welche die, in dieser Arbeit beschriebenen Verfahren, implementieren. }
\newcommand*{\IhreSchluesselwoerter}{Photogrammetrie, Computer Vision (CV), Augmented Reality (AR), standortbezogene Daten, Android, Java, Structure from Motion (SfM), Simultaneous Loaction and Mapping (SLAM)}


\usepackage[bookmarks, raiselinks, pageanchor, % PDF-Einstellungen
            hyperindex, colorlinks,
            citecolor=black, linkcolor=black,
            urlcolor=black, filecolor=black,
            menucolor=black]{hyperref}
\hypersetup{pdftitle={\IhrTitelDE},%
            pdfauthor={\IhrVorname\ \IhrNachname},%
            pdfsubject={\IhreArbeit},%
            pdfkeywords={\IhreSchluesselwoerter}}


%
% Beginn des Textteils
%
\begin{document}
  \pagenumbering{roman}
  \begin{titlepage}					% Titelseite
    \thispagestyle{empty}
    \begin{center}
      \Large
      Hochschule für Technik und Wirtschaft Berlin\\[1cm]
      \IhrStudiengang\\[1cm]
      \textbf{\IhreArbeit}\\[1cm]
      von\\[1cm]
      \IhrVorname\ \Sperren{\textbf{\IhrNachname}}\\[1cm]
      \textbf{\IhrTitelDE}\\[1cm]
      \IhrTitelEN
    \end{center}
  \end{titlepage}
  \clearpage
  \thispagestyle{empty}			% 1. Seite soll eine Leerseite sein (dazu muss ein Trick verwendet werden)
  \mbox{}
  \clearpage
  \thispagestyle{empty}			% 2. Seite wie Titelseite, aber mit zusaetzlichen Angaben
  \begin{center}
    \Large
    Hochschule für Technik und Wirtschaft Berlin\\
    Fachbereich Informatik, Kommunikation und Wirtschaft\\[1cm]
    Studiengang \IhrStudiengang\\[1cm]
    \textbf{\IhreArbeit}\\[1cm]
    von\\[1cm]
    \IhrVorname\ \Sperren{\textbf{\IhrNachname}}\\[1cm]
    \textbf{\IhrTitelDE}\\[1cm]
    \IhrTitelEN
  \end{center}
  \vspace*{5cm}
  \begin{tabbing}
    \underbar{Bearbeitungszeitraum:}\qquad\= von\qquad\=\IhrBearbeitungszeitraumVON\\
                                          \> bis      \>\IhrBearbeitungszeitraumBIS
  \end{tabbing}
  \vspace*{1cm}
  \underbar{1. Prüfer:}\qquad\IhrErstpruefer\par 
  \underbar{2. Prüfer:}\qquad\IhrZweitpruefer
  \clearpage
  % formblatt_para12apo.tex
%

\thispagestyle{empty}				% Formblatt Bestaetigung nach Paragraph 12 APO
\begin{minipage}{0.65\textwidth}
  Hochschule für Technik und Wirtschaft Berlin\\
  Fachbereich Informatik, Kommunikation und Wirtschaft\\[1.5cm]
\end{minipage}

Eigenständigkeitserklärung\\
\rule[1ex]{\textwidth}{0.5pt}
\vspace*{0.5cm}
\begin{tabbing}
  Name und Vorname\\
  der Studentin/des Studenten:\qquad\=\textbf{\IhrNachname, \IhrVorname}\\[1cm]
  Studiengang:                      \>\textbf{\IhrStudiengang}
\end{tabbing}
\rule[1ex]{\textwidth}{0.5pt}
\vspace*{0.5cm}
Ich bestätige, dass ich die \IhreArbeit\ mit dem Titel:
\begin{center}
  \textbf{\IhrTitelDE}
\end{center}
\vspace*{0.5cm}
selbständig verfasst, noch nicht anderweitig für Prüfungszwecke vorgelegt, keine anderen als die angegebenen Quellen oder Hilfsmittel benutzt, sowie wörtliche und sinngemäße Zitate als solche gekennzeichnet habe.\\[0.5cm]
\rule[1ex]{\textwidth}{0.5pt}
\begin{tabbing}
  Datum:\hspace{2cm}\=\today\\[1cm]
  Unterschrift:\> 
\end{tabbing}
\rule[1ex]{\textwidth}{0.5pt}
\clearpage		% 3. Seite: Formblatt Bestaetigung nach Paragraph 12 APO
  \include{formblatt_summary}			% 4. Seite: Formblatt Zusammenfassung
  \tableofcontents
  \newpage
  \pagenumbering{arabic}
  \onehalfspacing
  \chapter{Motivation}

Photogrammetrie ist "die Wissenschaft und Technologie der Gewinnung von Informationen
über die physische Umwelt aus Bildern, mit einem Schwerpunkt auf Vermessung,
Kartierung und hochgenauer Messtechnik". (Heipke, 2017, S.5 \cite{photo})  Die Photogrammetrie beschäftigt sich  mit der Rekonstruktion von dreidimensionalen Daten aus zweidimensionalen Informationsträgern, wie Bildern oder Laserscan Daten. Dabei gehen diese Daten alle auf das Prinzip der Aufnahme der elektromagnetischen Strahlung zurück. Bei Bildern ist das die Helligkeits und Farbverteilung, bei Laserscans, Entfernungsbilder, beziehungsweise Punktwolken. Die Disziplin der Photogrammetrie ist dabei dem Bereich der Fernerkundung zuzuordnen, die sich mit der Auswertung von geometrischen oder semantischen Informationen beschäftigt. Beides sind Fachbereiche, die sich über die Jahrzehnte entwickelt haben und sich dem Gebiet der Geodäsie zuordnen lassen. Die Geodäsie erfasst Geoinformationen über die Erde, die dann beispielsweise mit Kartographie visualisiert werden können. Das Gebiet der Computer Vision, das sich großteils parallel mit der Photogrammetrie entwickelt hat, verfolgt den gleichen Ansatz und ist inhaltlich sehr stark mit der digitalen Photogrammetrie verbunden. Das Forschungsziel der Computer Vision ist es, anhand von zeidimensionalen Bilddaten die geometrischen Informationen von dreidimensionalen Objekten, wie Form, Position, räumlicher Lage, Bewegung oder ähnliche Daten zu gewinnen. Unter diesem Gesichtspunkt gibt es sehr viele Gemeinsamkeiten zwischen Computer Vision und digitaler Photogrammetrie. (vgl. \cite{pose_est_epi} S.1) Nachdem sich Photogrammetrie und Computer Vision lange unabhängig voneinander entwickelt haben, ist Photogrammetrie heute als Grundlage von Computer Vision anerkannt. (vgl. \cite{photo} S. 5-7)

Durch die Entwicklung der Technik und der damit eingehenden Steigerung der Rechenpower im mobilen Bereich, haben sich die Bereiche, in denen Photogrammetrie eingesetzt werden kann vergrößert. Im Rahmen dieser Arbeit soll evaluiert werden, ob photogrammetrische Verfahren für Augmented Reality Anwendungen im Bereich von Smarthphones eingesetzt werden können, um in Echtzeit aus Videodaten die dreidimensionalen Beschaffenheit der gefilmten Objekte zu rekonstruieren. Dazu wird die photogrammetrische Pipeline analysiert und mit aktuellen Verfahren verglichen.

Im Rahmen dieser Arbeit ist ebenfalls eine auf Android basierende Anwendung erstellt worden, welche eine der neuen Technologien im Bereich Augmented Reality implementiert.



  



  \chapter{Augmented Reality}

Im Gegensatz zu \glqq Virtual Reality\grqq{} (VR), welche eine interaktive, dreidimensionale, computergenerierte, immersive Umgebung schafft, in die eine Person versetzt wird, erlaubt \glqq Augmented Reality\grqq{} (AR) die Überblendung von digitalen Medieninformationen über die Wahrnehmung der echten Welt. Dadurch fällt AR in die Definition von \glqq Mixed Reality\grqq{} (MR).

\begin{figure}[H]
	\centering
	\includegraphics[scale=0.52]{ar_vr.png}
	\caption{Das Realität - Virtualität Kontinuum, Bildquelle \cite{ar_vr}}
\end{figure} 

Im Rahmen dieser Arbeit wird sich, wenn der Begriff Augmented Reality verwendet wird, auf Monitor basierte, nicht immersive Geräte bezogen, da bei der Analyse von Photogrammetrie und Computer Vision sowie der Implementation einer AR Anwendung, das Smartphone als Medium im Fokus steht. Diese Anzeigesysteme werden auch als \glqq window-on-the-world\grqq{} bezeichnet, da computergenerierte Bilder oder Informationen digital über das wiedergegebene Echtzeit Kamera Bild des Smartphones überlagert werden. (vgl. \cite{ar_vr} S.284)



\section{Augmented Reality - SDKs}

\glqq Software Development Kits\grqq{} (SDK) oder auch Frameworks, sind Werkzeuge und Bibliotheken, welche Algorithmen und Basistechnologien liefern, um Programme zu entwickeln. Im Bereich von Augmented Reality umfassen Frameworks meistens die drei Hauptkomponenten (vgl. \cite{sdks} S.3):

\begin{itemize}

\item \textbf{Recognition}: Erkennung von Bildern, Objekten, Gesichtern oder Räumen, auf welche die virtuellen Objekte oder Informationen überlagert werden können.


\item \textbf{Tracking}: Echtzeit-Lokalisierung der erkannten Objekte und Berechnung der lokalen Position des Gerätes zu diesen.

\item \textbf{Rendering}: Überlagerung der virtuellen Medieninformationen über das Bild und Anzeige der generieren Mixed Reality.
\end{itemize}

\subsection{Marktübersicht - Software Development Kits}

Die folgende Tabelle gibt eine Übersicht an aktuellen und weit verbreiteten Software Development Kits, sowie deren Plattformkompatibilität zu Android und iOS: \\

\begin{table}[h!]
\hskip-1.5cm
\begin{tabular}{|l|l|l|l|l|l|l|l|l|l|}
\hline
        & Vuforia & Wikitude & Metaio & ARToolKit & Kudan & EasyAR & MaxST & ARCore & ARKit \\ \hline
Android &   \checkmark      &    \checkmark      &    \checkmark    &     \checkmark      &   \checkmark    &    \checkmark    &    \checkmark   &     \checkmark   &    x   \\ \hline
iOS     &    \checkmark      &   \checkmark        &   \checkmark      &    \checkmark        &   \checkmark    &    \checkmark     &   \checkmark     &    \checkmark     &    \checkmark    \\ \hline
Windows &     \checkmark     &    \checkmark       &     \checkmark    &    \checkmark        &   x    &     \checkmark    &  \checkmark      &    x    &    x   \\ \hline
\end{tabular}
\end{table} 


Bis auf das von Apple entwickelte ARKit, sind alle hier genannten Frameworks für Android verwendbar. Weiterhin unterstützen alle das Betriebsystem iOS, sowie Windows, bis auf Kudan, ARCore und ARKit. Im praktischen Teil dieser Arbeit werden mehrere dieser Software Development Kits auf der Android Plattform verwendet, getestet und analysiert.

\section{Voraussetzungen für Augmented Reality}
Augmented Reality Anwendungen haben hohe Anforderungen an die Rechnenpower der Technik, die Verarbeitungsgeschwindigkeit der Algorithmen und Robustheit der verwendeten Verfahren. Folgende Liste gibt eine Übersicht an wichtigen Aspekten um AR zu realisieren  (vgl. \cite{vorraussetzungen} S.1):


\begin{itemize}

\item \textbf{Hohe räumliche Genauigkeit}: 6 \glqq Degrees of Freedom\grqq{} (Freiheitsgrade) in Position und Ausrichtung. 

\item \textbf{Sehr geringer Jitter (Zittern)}: Das Rauschen im Tracking System muss minimal gehalten werden.

\item \textbf{Hohe Aktualisierungsraten}: mindestens 30Hz, besser mehrere 100Hz.

\item \textbf{Sehr geringer Lag}: Die Verzögerung von Messung bis zur Trackerausgabe muss minimal sein.

\item \textbf{Volle Mobilität}: Bewegungsfreiheit für den Nutzer: Keine Kabel, kein eingeschränkte Umfang an Bedienmöglichkeiten.
\end{itemize}

Erst durch die Entwicklung der Technik in den letzten zwei Jahrzehnten und einer damit eingehenden Steigerung der Rechenpower von mobilen Geräten, hat sich Augmented Reality auf Smartphones etablieren können.

\section{Arten des Augmented Reality Trackings}

Das Erkennen der Umgebung und die Lokalisierung der Kamera (Camera Pose Estimation) in dieser, siehe Abbildung 2.2, ist der ausschlaggebende Schritt zur Realisierung von Augmented Reality. Erst dieses Verfahren erlaubt die Projektion von digitalen Modellen in der richtigen Position auf den echten Bildern, um die Perspektiven von echter Welt und digitalem Modell zu vereinen. 


\begin{figure}[H]
	\centering
	\includegraphics[scale=0.58]{pose.png}
	\caption{Kamera Posenschätzung, anhand von Markern. Bildquelle \cite{pose}}
\end{figure} 


Präzise und robuste Kamerapositionsdaten sind eine Grundvoraussetzung für eine Vielzahl von Anwendungen, wie dynamischer Szenenanalyse und Interpretation, 3D-Szenen Strukturerkennung oder Videodatenkompression. Augmented Reality Umgebungen sind ein komplexes Anwendungsgebiet der Kameralokalisierung, da ein eingeschränkter Arbeitsbereich, limitierte Rechenpower und spezifische Anwendungsfälle hohe Anforderungen an die Robustheit und Schnelligkeit der Algorithmen stellen. 
Es existieren viele verschiedene Ansätze, um die Kameralokalisierung im Raum zu lösen. Das Problem wird als nichtlineares Problem betrachtet und wird meistens durch die \glqq Method of Least Squares\grqq{} (Methode der kleinsten Quadrate) oder  mit nichtlinearen Optimierungsalgorithmen gelöst, typischerweise durch das Gauß-Newton oder Levenberg-Marquardt Verfahren. (vgl. \cite{camera_pose} S.1) Im Folgenden werden die vier gängigsten Ansätze zur Lösung des Tracking Problems erläutert. 


\subsection{Referenzmarken-basiertes Tracking}

Markerbasiertes Tracking war lange Zeit eine der häufigsten verwendeten Techniken um Augmented Realtiy zu realisieren. Dies liegt in der einfachen Erkennung der Marker mit hohem Kontrast.

\begin{figure}[H]
	\centering
	\includegraphics[scale=0.6]{markers.png}
	\caption{Moderne Marker. Bildquelle \cite{markers}}
\end{figure} 

Für robuste Anwendungen werden meist schwarz-weiße Marker mit dicken Kanten verwendet, siehe Abbildung 2.3. Die Kanten dienen der Erkennung des Markers, das einzigartige Muster zur genauen Identifizierung, um welchen einzigartigen Marker es sich handelt (vgl. \cite{markers} S.2). Dadurch kann neben der Relation des Geräts zum Marker auch relativ einfach die Entfernung und der Winkel berechnet werden. Der Nachteil liegt in der Limitierung der Anwendungsgebiete, in denen diese Technik verwendet werden kann, da Marker immer im Sichtfeld der Kamera lokalisiert sein müssen und nicht von anderen Objekten verdeckt werden dürfen. Weiterhin müssen immer externe Ressourcen verwendet werden um diese Marker zu erstellen, zu registrieren und zu verwenden, was bei der Verwendung der Anwendung und damit der Nutzerfreundlicheit, immer mit einem Mehraufwand verbunden ist (vgl. \cite{comparative_sdks} S.13).

\subsection{Hybrid-basiertes Tracking}

Hybrid basiertes Tracking verwendet mehrere Datenquellen wie das Global Positioning System (GPS), den Kompass oder die Beschleunigungssensoren des Smartphones zur Bestimmung der Orientierung und Lokalisierung des Geräts. Dabei wird per GPS der Standort des Geräts bestimmt, um Objekte in der Nähe zu identifizieren, die augmentiert werden sollen. Mit Hilfe des Kompasses kann dann ein Pfad erstellt und überprüft werden, ob die Orientierung des Geräts auch in diese Richtung zeigt. Der Beschleunigungssensor bestimmt die Ausrichtung des Geräts mithilfe der Gravitation. Durch die Vereinigung all dieser Informationen aus den Sensoren, kann berechnet werden, was im Sichtfeld ergänzt werden soll, ohne dass eine Auswertung und Verarbeitung des realen aufgenommen zweidimensionalen Kamerabildes stattzufinden hat. Anschließend werden die Informationen über das Kamerabild gelegt.  (vgl. \cite{comparative_sdks} S.13, \cite{vorraussetzungen} S.4)

\subsection{Modell-basiertes Tracking}

Beim Modell-basiertem Tracking wird ein rekursiver Algorithmus verwendet. Hierbei wird die vorherige Kameraposition als Grundlage für die Berechnung der aktuellen Kameraposition verwendet. Dazu muss zuerst eine Kamerakalibrierung durchgeführt werden, anschließend wird die Pose des zu verfolgenden Objekts initial bestimmt. Dann wird bestimmt, welche Ecken des Objekts sichtbar sind. An diesen Ecken werden dann Kontrollpunkte verteilt, die in späteren rekursiven Aufrufen mit denen älterer Frames verglichen werden, was mit einer Karte der Ecken ermöglicht wird. Die Map der Ecken wird mithilfe des Canny-Algorithmus zur Eckenerkennung erstellt. Ist die Abweichung zum vorherigen Frame bestimmt worden, wird die Position der Kamera im Bezug zum Objekt aktualisiert. Durch die Rekursivität ist dieses Verfahren nicht sehr rechenintensiv und benötigt eine relativ geringe Prozessorleistung. Weiterhin kann zwischen verschiedenen Merkmalen unterschieden werden, welche für das Tracking verwendet werden. Bei der kantenbasierten Methode wird versucht ein dreidimensionales Wireframe mit den Kanten des Objekts in der realen Welt zuzuordnen. (vgl. \cite{model_based} S.2-3)

\begin{figure}[H]
	\centering
	\includegraphics[scale=0.75]{wire.png}
	\caption{Kantenbasiertes rekursives Tracking, Pipeline. Bildquelle: \cite{model_based} S.3}
\end{figure} 

Außerdem sind Ansätze wie \glqq Optical Flow based Tracking\grqq{}, was zeitliche Informationen, entnommen aus der Bewegung der Projektion des Objekts relativ zur Bildebene verwendet, sowie texturbasierte Ansätze verbreitet. (vgl. \cite{model_based} S.1)

\subsection{Natürliches Feature Tracking}

Natürliches Feature Tracking ist ein bildbasiertes Verfahren und kann die Position des Gerätes zur Umgebung bestimmen, ohne über Wissen über einen initialen vorherigen Zustand zu verfügen. Diese Methode ist in der Regel sehr rechenintensiv und benötigt hohe Prozessorleistung (vgl. \cite{model_based} S.1-2). Die Technik verwendet die Merkmale von Objekten in der echten Welt und erkennt ihre natürlichen Eigenschaften. Diese Merkmale werden \glqq Features\grqq{} genannt und sind typischerweise, basierend auf einem mathematischem Algorithmus, sehr gut unterscheidbar und außern sich in der Form von Ecken, Kanten oder starke Kontrasten. Die mathematischen Beschreibungen dieser Features (Deskriptoren) eines Bildes werden zur späteren Erkennung gespeichert. Anhand des gespeicherten Datensets aus Merkmalen kann dann erkannt werden, ob ein Bild im Vergleich zu einem anderen den gleichen Inhalt zeigt, unabhängig von Entfernung, Orientierung, Beleuchtungsintensität, Rauschen oder Verdeckung (vgl. \cite{comparative_sdks} S.13). Beim natürlichen Feature Tracking wird, meistens basierend auf dem Verfahren der kleinsten Quadrate und mit dem Gauß-Newton oder Levenberg-Marquardt Verfahren, versucht eine Reduzierung des \glqq Re-Projection Errors\grqq{} (Reprojektionsfehlers) zu erreichen, um alle internen und externen Kameraparameter zu bestimmen und zu optimieren. Dies sind nicht-lineare Methoden zur Lösung des Problems der kleinsten Quadrate. Typischerweise sind zwei bis vier Wiederholungen genug. (vgl. \cite{natural_feature} S.28-29) Die genaue Funktionsweise des Natürlichen Feature Trackings wird in Kapitel 3.5 und 3.6 weiter ausgeführt.
 

\section{Photogrammetrie oder Computer Vision für AR}

In Kapitel 2 wurde eine kurze Einführung in Augmented Reality gegeben und deren Voraussetzungen sowie aktuellen Umsetzungen der Kameralokalisierung vorgestellt. Im folgenden Teil dieser Arbeit wird Photogrammetrie, sowie Verfahren aus der Computer Vision, wie etwa SLAM (Simultaneous Localisation and Mapping), welches von den meisten Augmented Reality Software Development Kits heutzutage verwendet wird, beschrieben. Weiterhin wird ein Vergleich der beiden Verfahren durchgeführt, um Gemeinsamkeiten, Unterschiede, sowie Möglichkeiten und Schwächen der einzelnen Verfahren aufzuzeigen und in Kontext zu bringen. Anschließend wird evaluiert ob photogrammetrische Verfahren in Kontext der Augmented Reality eingesetzt werden können.


  \chapter{Photogrammetrie}

\section{Einführung in die Photogrammetrie}

Das Grundprinzip der Messung mit Kameras ist einfach. Licht breitet sich mit einer bestimmten Wellenlänge, in annähernd geraden Strahlen aus. Diese Strahlen werden vom Sensor der Kamera aufgenommen, sodass diese die Richtungen im dreidimensionalen Raum misst. Der grundlegende geometrische Zusammenhang der Photogrammetrie ist somit die Zentralprojektion, die sich mathematisch durch die Kollinearitätsgleichung beschreiben lässt. Ein dreidimensionaler Punkt in der echten Welt, sein Bild in der Kamera und das Projektionszentrum müssen alle auf einer geraden Linie liegen. (vgl. \cite{fiundations_pg} S.1) Das fundamentale photogrammetrische Problem besteht in der Bestimmung von internen und externen Ausrichtungsparametern der Kamera und der  Messung von Objekt und Raumkoordinaten der aufgenommenen Fotografien. 


\begin{itemize}
\item \textbf{Interne Orientierung}: Bei der internen Orientierung werden Kameraparameter gemessen und ausgewertet. Dazu wird die \glqq principle distance\grqq{} (Brennweite) und der \glqq principle point\grqq{} (Optisches Zentrum) betrachtet.

\begin{figure}[H]
	\centering
	\includegraphics[scale=0.45]{pp.png}
	\caption{Kamera Kalibierungsmodell, Bildquelle \cite{pp}}
\end{figure} 

Weiterhin müssen Parameter, welche die Verzeichnung, also die nicht maßstabsgetreue Abbildung von Objekten, betrachtet werden. Diese Parameter, die beispielsweise in der Objektivkorrektur verwendet werden, müssen, um die interne Orientierung der Kamera genau abzubilden, mit in die Berechnung einfließen.

\item \textbf{Externe Orientierung}: Bei der externen Orientierung wird versucht die genaue räumliche dreidimensionale Lage der Kamera zum Zeitpunkt der Belichtung des Bildes zu rekonstruieren. Für die Bestimmung der Orientierung von ein oder mehreren Fotos, können verschiedene Methoden verwendet werden. Dies kann in Teilschritten (relative und absolute Orientierung) oder gleichzeitig (Bündelblockausgleich) durchgeführt werden. (vgl. \cite{exterior_review} S.616)
\end{itemize}

Khalid El-AShmawy \cite{comparative_conditions_study} beschreibt die Verwendung von Strahlenbündel, die durch Fotos generiert werden, als zweifelsfrei den flexibelsten Ansatz zur Blockbildung, Blockanpassung und für Photogrammetrie im Allgemeinen und mit den besten Ergebnissen. In der Nahbereichsphotogrammetrie, bei der mehrstufige und konvergente Konfigurationen möglich sind, ist der Bündelansatz in seiner stärksten Form vertreten. 


Der Ausgleich der Strahlenbündel in einem Set an Fotos beinhaltet die Rotation und Translation von jedem Bündel im Raum in eine Position, in der alle Strahlen sich an der korrekten Position im Objektraum schneiden. (vgl. \cite{comparative_conditions_study} S.66)

\section{Bündelblockausgleich}

Das Verfahren des Bündelblockausgeleichs, verwendet die Methoden der \glqq collinearity condition \grqq{} (Kollinearitätsbedingung), der \glqq coplanarity condition\grqq{} (Koplanaritätsbedingung) oder die Methode der direkten linearen Transformation. Die gewünschten Parameter aller Fotos werden gleichzeitig durch eine iterative Wiederholung der \glqq least square\grqq{} Methode (Methode der kleinsten Quadrate) angepasst und korrigiert. Die Iterationen sind durch die nicht-Linearität der Konditionsgleichungen notwendig. Die Resulate des Bündelblockausgleichs aller Fotos sind dann die Ergebnisse der externen Orientierung der Kamera für jedes einzelne Foto. Weiterhin ergibt sich eine Auflistung der Objektraumkoordinaten der gemessenen Punkte aller Fotos, sowie deren gemessene statische Genauigkeit. (vgl. \cite{comparative_conditions_study} S.66-67)

\url{http://sci-hub.tw/https://doi.org/10.3846/20296991.2015.1051335}
\url{http://sci-hub.tw/10.1111/0031-868X.00210}

Neun Elemente:
\url{http://sci-hub.tw/10.1111/phor.12037}

Gutes Paper:
\url{file:///C:/Users/Daniel/Downloads/qu2018msc.pdf}

\section{Nicht-lineare Methode der kleinsten Quadrate}
Die \glqq Method of Least Squares\grqq{} (Methode der kleinsten Quadrate) ist eine sehr leistungsfähige und flexible Technik, um alle Arten von Datenabgleichsproblemen zu lösen. Das Verfahren wird eingesetzt, um eine parametrisierbare Funktion an ein Datenset von Messwerten anzupassen, durch Minimierung der Summe des quadratischen Fehlers zwischen Funktion und Datenpunkten. Die verschiedenen Werkzeuge dieser Methode können auch eingesetzt werden, um beispielsweise die Korrelationsqualität der Messdaten zu bewerten. Gleichzeitig ermöglicht das System die Stabilisierung und Verbesserung der Korrelation, durch die Berücksichtigung geometrischer Randbedingungen. Wenn die anzupassende Funktion nicht linear ist, ist das Problem der kleinsten Quadrate ebenfalls nicht linear. Nichtlineare Lösungen der Methode der kleinsten Quadrate reduzieren iterativ die Summe der quadratischen Fehler zwischen Funktion und Messwerten, durch eine Abfolge von Aktualisierungen der Parameterwerte. (vgl. \cite{least_quares} S.1, \cite{lev_mar} S.1)

In den folgenden Abschnitten werden verschiedene Methoden zur Bestimmung der externen Kameraparameter während des Bündelblockausgleichs vorgestellt. Die Methode der kleinsten Quadrate ist dabei essentiell für die Anwendbarkeit dieser Algorithmen, weswegen diese sowie das Gauss-Newton und das Levenberg-Marquardt Verfahren vorgestellt wird. 

Die nicht lineare Methode der kleinsten Quadrate ist ein Standardoptimierungsproblem und wird definiert als \cite{nonlinear_1} :

\begin{equation}
minimiere: \sum_{i=1}^m f_i(x)^2 =  ||f(x)||^2
\end{equation} 

Wobei: 

$f_1(x),...,f_m(x)$ differenzierbare Funktionen einer Vektorvariablen $x$ sind.

$f$ eine Funktion von $\textbf{R}^n$ zu $\textbf{R}^m$ mit den Komponenten $f_i(x)$ ist:

\begin{equation}
f(x) = \begin{bmatrix}
f_1(x)\\ f_2(x)\\ \cdots \\ f_m(x)
\end{bmatrix}
\end{equation} 

Das Ziel ist es ein $x$ zu finden, dass $||f(x)||^2$ minimiert. In unserem Fall, der Positionsbestimmung von Objekten im Raum mit der Grundlage mehrerer Kamerabilder, kann $f$ folgendermaßen aufgestellt werden:

Das \textbf{Kameramodell} wird beschrieben durch die Parameter: $a \in  \textbf{R}^{2\times 3}, b \in \textbf{R}^2, c \in  \textbf{R}^3, d \in \textbf{R}$ welche die Position und Orientierung charakterisieren. Ein Objekt an Position $x \in \textbf{R}^3 $ erzeugt ein Bild an der Position $x^\prime \in \textbf{R}^2$ auf der Bildebene.

\begin{equation}
x \prime = \frac{1}{c^Tx+d} (Ax +b )
\end{equation}

$c^Tx+d >0$, wenn das Objekt vor der Kamera ist. Angenommen ein Objekt an Position $x_{ex}$ wird von $l$ Kameras betrachtet. (Beschrieben durch $a_i,b_i,c_i,d_i$) Das Bild des Objekts in der Bildebene von Kamera $i$ ist an Position:

\begin{equation}
y_i = \frac{1}{c_i^T x_{ex}+d_i}(A_i x_{ex} + b_i) + v_i
\end{equation}

Wobei $v_i$ der Mess- oder Quantisierungsfehler ist. Das Ziel ist es nun die dreidimensionale Position $e_{ex}$ von den $l$ Beobachtungen $y_1,...,y_l$ zu schätzen. 

\textbf{Nicht-lineare Methode der kleinsten Quadrate:} Berechne die Schätzung von $\hat x$ durch die Minimierung von (vgl. \cite{nonlinear_1} S.5-6):

\begin{equation}
\sum_{i=1}^l || y_i = \frac{1}{c_i^T x_{ex}+d_i}(A_i x_{ex} + b_i) + v_i||²
\end{equation}

\subsection{Gauß-Newton-Verfahren}

Das Gauß-Newton Verfahren ist eine Technik zur Lösung der nicht linearen Methode der kleinsten Quadrate. Das Verfahren besteht aus einer Folge von linearen Annäherungen der kleinsten Quadrate an das nicht lineare Problem, bei dem jedes einzelne durch einen direkten oder iterativen Prozess gelöst wird. (vgl. \cite{approx_gn} S.1) Gegeben ist die Definition der Problemstellung der kleinsten Quadrate, siehe (3.1). Beginnend mit einer anfänglichen Schätzung $x^{(1)}$, werden weitere Näherungslösungen $k = 1,2,...$ berechnet (vgl. \cite{nonlinear_1} S.16-17).

Dazu wird $f$ um $x^{(k)}$ linearisiert:

\begin{equation}
\overline{f}(x;x^{(k)}) = f(x^{(k)})+Df(x^{(k)})(x-x^{(k)})
\end{equation}

Ersetze die affine Annäherung $ \overline{f}(x;x^{(k)})$ für $f$ im Problem der kleinsten Quadrate (3.1):
\begin{equation}
minimiere: ||\overline{f}(x;x^{(k)})||²
\end{equation}

Die Lösung dieses linearen Problems wird nun als $x^{(k+1)}$ verwendet. 

Das Problem der kleinsten Quadrate ist in Wiederholung $k$ des Gauß-Newton Verfahrens gelöst:
\begin{equation}
minimiere: ||f(x^{(k)}) + Df(x^{(k)})(x-x^{(k)})||²
\end{equation}

Wenn $Df(x^{(k)}) $ linear unabhängige Spalten hat, wird die Lösung gegeben durch:
\begin{equation}
x^{k+1} = x^{(k)}-\Big(Df(x^{(k)})^TDf(x^{(k)})\Big)^{-1} Df(x^{(k)})^T f(x^{(k)})
\end{equation}

Der Gauß-Newton Schritt $\Delta x^{(k)} = x^{(k+1)} - x^{(k)}$ ist:
\begin{equation}
\begin{aligned}
\Delta x^{(k)} &= -\Big(Df(x^{(k)})^TDf(x^{(k)})\Big)^{-1} Df(x^{(k)})^T f(x^{(k)})\\ &= -\frac{1}{2} \Big(Df(x^{(k)})^TDf(x^{(k)})\Big)^{-1} \nabla g(x^{(k)})
\end{aligned}
\end{equation}

Wobei $\nabla g(x)$ der Gradient der Kosten für nichtlineare kleinste Quadrate ist.

Das Gauß-Newton Verfahren eignet sich besonders gut für die Verarbeitung von großen Datenmengen mit hoher Varianz. Im Vergleich zum Newton-Verfahren ist der Algorithmus attraktiver, da er keine Auswertung der zweiten Ableitungen in der Hesse-Matrix der Zielfunktion benötigt. (vgl.\cite{approx_gn} S.1, \cite{nonlinear_1} S.16-17)


\subsection{Levenberg-Marquardt Methode}

Die Levenberg-Marquardt Methode wurde 1944 von Kenneth Levenberg \cite{levenberg} veröffentlicht, in den 1960er Jahren von Donald Marquardt wiederentdeckt und wurde entwickelt um nicht lineare Probleme der kleinsten Quadrate zu lösen. Der Algorithmus kombiniert zwei Minimierungsmethoden: \glqq Gradient descent\grqq{} (Gradientenverfahren) und das Gauß-Newton Verfahren. Beim Gradientenverfahren wird die Summe der quadratischen Fehler durch die Aktualisierung der Parameter in Richtung der steilsten Richtung zum Minimum hin reduziert. Das Levenberg-Marquardt Verfahren verhält sich wie das Gradientenverfahren, wenn die Parameter weit von ihrem optimalen Wert entfernt sind und wie das Gauß-Newton Verfahren, wenn die Parameter nahe an ihrem optimalen Wert liegen. Es wechselt also adaptiv die Parameterupdates zwischen den beiden Verfahren. (vgl. \cite{lev_mar} S.1)

Der Algorithmus befasst sich mit zwei Problembereichen des Gauß-Newton Verfahrens:

\begin{itemize}
\item Wie aktualisiert man $x^{(k)}$, wenn die Spalten von $Df(x^{(k)})$ linear Abhängig sind.

\item Wie verfährt man, wenn das Gauß-Newton Update $||f(x)||²$ nicht reduziert.
\end{itemize}

Das Verfahren wird mathematisch wie folgt beschrieben:

Berechnung von $x^{(k+1)}$ durch Lösung eines normalisierten Problems der kleinsten Quadrate:
\begin{equation}
minimiere: ||\overline{f}(x;x^{(k)})||² + \lambda^{(k)}||x-x^{(k)}||²
\end{equation}

$\overline{f}(x;x^{(k)})$ ist dabei wie in Gleichung 3.6 gegeben. Mit $\lambda^{(k)} > 0$ gibt es immer eine eindeutige Lösung.

Das normalisierte Problem der kleinsten Quadrate wird in Iteration $k$ gelöst:
\begin{equation}
minimiere: ||f(x^{(k)}) + Df(x^{(k)})(x-x^{(k)})||² + \lambda^{(k)}||x-x^{(k)}|²
\end{equation}

Die Lösung ist gegeben durch:
\begin{equation}
x^{(k+1)} = x^{(k)} -\Big(Df(x^{(k)})^TDf(x^{(k)})+\lambda^{(k)}I\Big)^{-1} Df(x^{(k)})^T f(x^{(k)})
\end{equation}

Der Levenberg-Marquardt Schritt $\Delta x^{(k)} = x^{(k+1)} - x^{(k)}$ ist:
\begin{equation}
\begin{aligned}
\Delta x^{(k)} &= -\Big(Df(x^{(k)})^TDf(x^{(k)})+\lambda^{(k)}I\Big)^{-1} Df(x^{(k)})^T f(x^{(k)})\\ &= -\frac{1}{2} \Big(Df(x^{(k)})^TDf(x^{(k)})+\lambda^{(k)}I\Big)^{-1} \nabla g(x^{(k)})
\end{aligned}
\end{equation}

Für $\lambda^{(k)}=0$ ist das der Gauß-Newton Schritt; für große $\lambda{(k)}$:
\begin{equation}
\Delta x^{(k)} \approx -\frac{1}{2}\lambda^{(k)}\nabla g(x^{(k)})
\end{equation}

Es gibt verschiedene Strategien um $\lambda^{(k)}$ anzupassen:

\begin{itemize}
\item Nach Iteration $k$, berechne Lösung $\hat{x}$ von Gleichung (3.11)
\item Wenn $||f(\hat{x})||²<||f(x^{(k)})||²$, verwende $x^{(k+1)} = \hat{x}$ und verringere $\lambda$
\item Wenn $||f(\hat{x})||²>||f(x^{(k)})||²$, lasse $x$ gleich (verwende $x^{(k+1)} = x^{(k)}$), und erhöhe $\lambda$
\end{itemize}
(vgl. \cite{nonlinear_1} S.19-21)


\subsection{Bestimmung der externen Kameraparameter mit der Kollinearitätsbedingung}

Die gleichzeitige Anpassung verwendet die Kollinearitätsbedingung  um zwei Gleichungen für jeden gemessenen Bildpunkt aufzustellen. Die Lösung all dieser Gleichungen erfolgt dann nach der Methode der kleinesten Quadrate. Die Bedingung der Kollinearität sagt aus, dass ein Objektpunkt $P$, sein Bildpunkt $p$ und das perspektivische Zentrum $O$, alle auf der gleichen Geraden liegen müssen. Mathematisch wird die Bedingung wie folgt ausgedrückt  \cite{coll_exterior}:

\begin{equation}
\begin{aligned}
  x_p = -f \frac{(X_p-X_O)m_{11}+(Y_p-Y_O)m_{12}+(Z_p-Z_O)m_{13}}{(X_p-X_O)m_{31}+(Y_p-Y_O)m_{32}+(Z_p-Z_O)m_{33}} \\
    y_p = -f \frac{(X_p-X_O)m_{21}+(Y_p-Y_O)m_{22}+(Z_p-Z_O)m_{23}}{(X_p-X_O)m_{31}+(Y_p-Y_O)m_{32}+(Z_p-Z_O)m_{33}}
\end{aligned}
\end{equation}

Dabei sind $x_p$ und $y_p$ die korrigierten Foto Koordinaten, $X_p,Y_p,Z_p$ die Objektpunktkoordinaten von $P$, $X_O,Y_O,Z_O$ die Koordinaten des perspektivischen Zentrums $O$, $f$ die kalibrierte Brennweite der Kamera und $m_{ij}$ die Elemente der Orientierungsmatrix $M$ des Fotos.

Die linearisierte Form der Gleichung, für die Lösung der Methode der kleinsten Quadrate, wird gegeben als (\cite{comparative_conditions_study} S.67):

\begin{equation}
V+B\cdot\triangle =\varepsilon
\end{equation}

Wobei:
\begin{itemize}
\item $\triangle$ der Korrekturvektor zu dem aktuellen Werteset, für die unbekannten Werte (innere und äußere Orientierung, Objektkoordinaten der Punkte) der iterativen Lösung ist.

\item $B$ die Matrix der partiellen Ableitungen von Gleichung (3.1), in Bezug auf die Unbekannten  ist.

\item $V$ der Korrekturvektor zu den Beobachtungen ist.

\item $\varepsilon$ der Abweichungsvektor ist.
\end{itemize}

El-Ashmawy \cite{comparative_conditions_study} schlägt weitere Beschränkungen vor, indem ergänzende Beobachtungsgleichungen berücksichtigt werden, die sich aus den a priori Kenntnissen der Raumkoordinaten der Objekte der Kontrollpunkte in Gleichung (3.2) ergeben. Solche zusätzlichen Gleichungen können wie folgt beschrieben werden:

\begin{equation}
V^c-\triangle^c = \varepsilon^c
\end{equation}

Wobei:

\begin{itemize}
\item $\triangle^c$ der Vektor der beobachtbaren Korrekturen zu den Objektkoordinaten der Kontrollpunkte ist.

\item $\varepsilon^c$ der Abweichungsvektor zwischen Beobachtungswerten und aktuellen (in iterativer Lösung) Werten der Objektkoordinaten der Kontrollpunkte ist.

\end{itemize}

Beobachtungsgleichungen können dann durch das Zusammenführen von Gleichung (3.2) und (3.3) erhalten werden. Die grundlegenden Voraussetzungen an den Bündelblockausgleich sind die Schätzungen der Parameter für innere und äußere Orientierung der Kamera. Weiterhin können die - je nach spezifischem Ansatz - Schätzungen für die Objekt und Raumkoordinaten aller Kontrollpunkte  nützlich sein. Deshalb sollte ein Bündelverfahren immer eine praktikable Methode enthalten, mit der die ungefähren geschätzten initialen Werte ermittelt werden können. Dieses Vorwissen sorgt nicht nur für eine reduzierte Anzahl an Iterationen, sondern resultiert auch in schnelleren und genaueren Ergebnissen. (vgl. \cite{comparative_conditions_study} S.67)


\subsection{Bestimmung der externen Kameraparameter mit der Koplanaritätsbedingung}

Die Koplanaritätsbedingung impliziert, dass die beiden perspektivischen Zentren von zwei Aufnahmen, ein beliebiger Objektpunkt und die entsprechenden Bildpunkte auf den beiden Fotos, alle in einer gemeinsamen Ebene liegen müssen (vgl. Abbildung 3.2). Die Koplanaritätsbedingung kann wie folgt dargestellt werden (\cite{pose_est_epi} S.1204):

\begin{equation}
F_i = \begin{vmatrix}
b_X & b_Y & b_Z \\
X_1 & Y_1 & Z_1 \\
X_2 & Y_2 & Z_2
\end{vmatrix}
=0
\end{equation}

Dabei sind $b_X,b_Y,b_Z$ die Komponenten des Basisvektors $b$ und $X_1,Y_1,Z_1$ sowie $X_2,Y_2,Z_2$ sind die Komponenten des Vektors $\vec{R_1}$ (von $O_1$ zu $P$) respektive  $\vec{R_2}$ (von $O_2$ zu $P$).


\begin{figure}[H]
	\centering
	\includegraphics[scale=0.6]{coplanarity.png}
	\caption{Koplanaritätsbedinung, Bildquelle \cite{comparative_conditions_study}}
\end{figure} 

Das mathematische Modell besteht aus vier skalaren Gleichungen:

\begin{equation}
\begin{aligned}
X_p - (X_{O_1}+0.5(b_X+ \lambda \cdot X_1 + p \cdot X_2)) = 0.0 \\
Y_p - (Y_{O_1}+0.5(b_Y+ \lambda \cdot Y_1 + p \cdot Y_2)) = 0.0 \\
Z_p - (Z_{O_1}+0.5(b_Z+ \lambda \cdot Z_1 + p \cdot Z_2)) = 0.0 \\
D_Y = \lambda\cdot X_1-p\cdot X_2-b_Y = 0.0
\end{aligned}
\end{equation}

Wobei $X_{O_1}, Y_{O_1},Z_{O_1}$ die Objektkoordinaten der ersten Kameraposition während der Belichtung sind und $\lambda$ und $p$ die Skalierungsfaktoren der entsprechenden Positionsvektoren $\vec{r_1}$ und $\vec{r_2}$ im Kameraraum sind. 

Die linearisierte Form von Gleichung (3.5), mit ebenfalls von El-Ashmawy \cite{comparative_conditions_study} vorgeschlagenen zusätzlichen Beschränkungen, für das Verfahren der kleinsten Quadrate, wird gegeben als:

\begin{equation}
\begin{aligned}
A\cdot V + B\cdot \triangle = \varepsilon \\
V^c - \triangle^c = \varepsilon^c
\end{aligned}
\end{equation}


Wobei:
\begin{itemize}
\item $\triangle$ der Korrekturvektor zu dem aktuellen Werteset, für die unbekannten Werte (innere und äußere Orientierung, Objektkoordinaten der Punkte) der iterativen Lösung ist.

\item $A$ die Matrix der partiellen Ableitungen von Gleichung (3.5), in Bezug auf die Beobachtungen (korrigierte Foto-Koordinaten auf den linken und rechten Fotos, des gleichen Objektpunkts) ist.

\item $B$ die Matrix der partiellen Ableitungen von Gleichung (3.5), in Bezug auf die Unbekannten  ist.

\item $V$ der Korrekturvektor zu den Beobachtungen ist.

\item $\varepsilon$ der Abweichungsvektor ist.
\end{itemize}

Zur Verwendung der iterativen Lösung der kleinsten Quadrate, ist die Berechnung der Ausgangswerte von Unbekannten, wie beim Verfahren mit der Kollinearitätsbedinung, notwendig. (vgl. \cite{comparative_conditions_study} S.68)


\subsection{Bestimmung der externen Kameraparameter mit der Direct Linear Transformation Methode}

Das \glqq direkte lineare Transformationsverfahren\grqq{} (DLT) modelliert die Transformation zwischen Bildkoordinatensystem und Objektkoordinatensystem als lineare Funktion und wurde von Abdel-Aziz und Karara \cite{dlt_intro} eingeführt. Die direkte lineare Transformation kann aus den Kollinearitätsgleichungen abgeleitet werden und lassen sich mathematisch wie folgt ausdrücken (\cite{dlt} S.72)

\begin{equation}
\begin{aligned}
x=\frac{L_1X+L_2Y+L_3Z+L4}{L_9X+L_{10}Y+L_{11}Z+1} \\
y=\frac{L_5X+L_6Y+L_7Z+L_8}{L_9X+L_{10}Y+L_{11}Z+1}
\end{aligned}
\end{equation}

Wobei $x,y$ die Bildkoordinaten, $L_1,...,L_{11}$ die Transformationskoeffizienten und $X,Y,Z$ die Objektkoordinaten des Punktes sind. Die Werte der inneren und externen Kameraparameter werden dann berechnet durch:

\begin{equation}
\begin{bmatrix}
X_0 \\ Y_0 \\ Z_0 
\end{bmatrix}
 = -
 \begin{bmatrix}
 L_1 & L_2 & L_3 \\
 L_5 & L_6 & L_7 \\
 L_9 & L_{10} & L_{11}
 \end{bmatrix}^{-1}
 \begin{bmatrix}
 L_4 \\ L_8 \\ 1.0
 \end{bmatrix}
 \end{equation}
 
 \begin{equation}
 \begin{aligned}
 x_0 &= (L_1L_9 + L_2L_{10} + L_3L_{11})/(L²_9 + L²_{10} + L²_{11}); \\
 y_0 &= (L_5L_9 + L_6L_{10} + L_7L_{11})/(L²_9 + L²_{10} + L²_{11}); \\
 \omega &= tan^{-1}(-L_{10}/L_{11}); \\
 \phi &= sin^{-1}(-L_9 \sqrt{(L²_9 + L²_{10} + L²_{11})} \\
  \kappa &= cos^{-1}((x_0L_9 - L_1)/(cos \phi \sqrt{(x_0L_9-L_1)² + (x_0L_{10}-L_2)² + (x_0L_{11}-L_3)²})); \\
 f&=(x_0L_9)/(cos \kappa \cdot \phi \sqrt{L²_9 + L²_{10} + L²_{11}}) 
 \end{aligned}
 \end{equation}

Wobei $X_0,Y_0,Z_0,\omega ,\phi , \kappa $ die externen Kameraparameter, $x_0,y_0$ die Bildkoordinaten des optischen Zentrums und $f$ die Brennweite ist. 

Das direkte lineare Transformationsverfahren hat lange Zeit in den Bereichen Photogrammetrie, Computer Vision, Robotik und Biomechanik Verwendung gefunden. Dies liegt an der linearen Formulierung der Beziehung zwischen Objekt- und Bildkoordinaten. Weiterhin können Bildkoordinaten in einem nicht-orthogonalem System, mit unterschiedlichen Skalen ausgedrückt werden und die Position des Koordinatensystems kann unbekannt, sowie die Brennweite beliebig sein und von Bild zu Bild variieren. (vgl. \cite{dlt} S.72)



\subsection{Feature Matching}

\subsection{Image Matching}

state of art source

\subsection{Structure from Motion}

6.2 Structure from motion

\url{http://sci-hub.tw/https://doi.org/10.1007/s10462-012-9365-8}

\subsection{Depth Maps}

\section{Photogrammetrie für Smartphones}

\section{Evaluierung der photogrammetrischen Technologie für Echtzeit AR Anwendungen}
  \chapter{Simultaneous Localisation and Mapping}

Simultaneous Localisation and Mapping, kurz SLAM, ist das Problem der Auswertung einer unbekannten Umgebung und Erstellung einer Map, während gleichzeitig die lokale Position innerhalb dieser Map bestimmt wird. Die Lösung dieses SLAM Problems war vorallem in der Robotik eine fundamentale Aufgabe der letzten zwei Jahrzehnte. Dabei ist SLAM ein Alltagsproblem: Das Problem der räumlichen Erkundung. Jeder Mensch und jedes Tier hat dieses Verfahren gemeistert und benutzt es unterbewusst zur Navigation in unserer Realität. Die Lösung dieses Problems, wenn es für einen Roboter automatisiert ausgeführt werden soll, ist dagegen sehr komplex. Durch das Meistern dieser Technik kann man Roboter wirklich autonom steuern. 
Bei SLAM wird die Bewegung des Objekts an sich durch den Raum und die Position aller zur positionsbestimmung notwendigen Merkmale berechnet, ohne auf vorheriges Wissen, über Position oder Lage im Raum, Kenntniss zu haben. (vgl. \cite{slam} S. 1-2) 

Dabei benötigt der Roboter mindestens einen exterozeptiven Sensor um äußere Informationen zu sammeln.
SLAM besteht aus drei grundlegenden Operationen, die iterativ pro Zeitintervall ausgeführt werden.

\textbf{Der Roboter bewegt} sich und erreicht eine neue Position in der Umwelt. Diese Bewegung erzeugt, durch unvermeidbares Rauschen und Fehler, Ungewissheit über die wirkliche Position des Roboters. Eine automatisierte Lösung benötigt ein mathematisches Modell für diese Bewegung. Dies ist das \glqq\textit{Motion Model}\grqq

\textbf{Der Roboter entdeckt neue Features} in seiner Umgebung, welche in die Umgebungskarte aufgenommen werden müssen. Diese Features heißen \glqq Landmarks\grqq . Da die Position der Landmarks, durch Fehler in den exterozeptiven Sensoren und die Position des Roboters ungewiss ist, müssen diese beiden Faktoren passend arrangiert werden. Eine automatisierte Lösung benötigt ein mathematisches Modell, das die Position der Landmarks anhand er Sensordaten bestimmt. Dies ist das \glqq \textit{Inverse Oberservation Model} \grqq .

\textbf{Der Roboter entdeckt Landmarks, die schon gemappt wurden} und verwendet diese um seine eigene Position, sowie die aller Landmarks zu korrigieren. Diese Operation reduziert die Unsicherheit über den Standort des Roboters, sowie der Landmarks. Die automatisierte Lösung erfordert ein mathemathisches Modell, um die Werte der Messungen aus den prognostizierten Positionen der Landmarks und der Position des Roboters zu berechnen. Dies ist das \glqq \textit{Direct Observation Model} \grqq

Mit diesen drei Modellen ist es möglich eine automatisierte Lösung für SLAM zu entwerfen. Diese Lösung muss diese drei Modelle verbinden und alle Daten korrekt und oganisiert halten, sowie die korrekten Entscheidungen bei jedem Schritt machen. (vgl. \cite{ekf_slam} S.2-3)

Eine erfolgreiche Lösung des SLAM Problems setzt weiterhin die Lösung des \glqq Loop Closure Detection\grqq Problems vorraus. Dabei müssen bereits besuchte Orte in der beliebig großen Map erkannt werden. Wegen der möglichen Komplexität von großen Maps ist es auch eins der größten Hindernisse, wenn es um die Skalierbarkeit der Lösung geht. Wichtig ist, dass die Loop Closure Detection keine Falsch-Positiven Ergebnisse liefert, da dies die Integrität und Korrektheit der kompletten Map beeinflusst. (vgl. \cite{ar_slam} S.4)


\begin{figure}[H]
	\centering
	\includegraphics[scale=0.35]{slam_problem.png}
	\caption{Das SLAM Problem: Die wahren absoluten Positionen der extrahierten Features sind nie wirklich bekannt. Bildquelle \cite{slam}}
\end{figure} 

Wie in Abbildung 3.1. erkennbar ist, bewegt sich ein Roboter durch eine unbekannte Umgebung und nimmt mit seinem Sensor Features der näheren Objekte (Landmarks) auf. Wobei \large\textbf{x}\normalsize\textit{k} der Vektor des Roboters,  \large\textbf{u}\normalsize\textit{k} der Bewegungsvektor, \large\textbf{m}\normalsize\textit{i} der Vektor des Landmarks und \large\textbf{z}\normalsize\textit{ik} die Oberservation eines Landmarks durch den Roboter zur Zeit \large\textit{k }\normalsize sind. Wie man sehen kann, ist der Fehler zwischen echten und geschätzten Landmarks, bei allen geschätzten Landmarks ähnlich, was an der initialen Betrachtung der Umgebung liegt. Zu diesem Zeitpunk wird nur das erste Feature erkannt. Daraus kann man schließen, dass die Fehler in der Schätzung der Landmarkpositionen korrelieren. Praktisch bedeutet dies, dass die relative Position zweier Landmarks, \large\textbf{m}\normalsize\textit{i} - \large\textbf{m}\normalsize\textit{j} zueinander sehr genau sein kann, auch wenn die absolute Position sehr ungenau ist. 

Je mehr Landmarks in das Modell aufgenommen werden, desto gleichbleibend besser wird das Modell der relativen Positionen, egal wie sich der Roboter bewegt. Dieser Prozess wird in Abbildung 3.2. veranschaulicht.

\begin{figure}[H]
	\centering
	\includegraphics[scale=0.5]{slam_springs.png}
	\caption{Die Landmarks sind durch Federn verbunden, welche die Korrelation zwischen ihnen darstellen.  Bildquelle \cite{slam}}
\end{figure}  

Während sich der Roboter durch die Umgebung bewegt, werden die Korrelationen stetig aktualisiert. Je mehr Beobachtungen über die Umwelt gemacht werden, desto steifer werden die Federn in diesem Modell. Im Nachhinein werden neue Beobachtungen von Landmarks durch das ganze Netzwerk propagiert und je nach Input, kleinere oder größere Anpassungen vorgenommen.

Lösungen für das SLAM Problem benötigen eine angemessene Repräsentation für die Observierungen der Landmarks, welche eine konsistente und schnelle Berechnung ermöglichen. Die geläufigste Repräsentation besteht in der Form einer Zustandsraumdarstellung mit Gaußschen Rauschen, was zur Verwendung des \glqq Extended Kalman Filter\grqq (EKF) führt. (vgl. \cite{slam} S. 2-4)

Weitere gängige Lösungen für das SLAM Problem sind \glqq Maximum Likelihood Techniques \grqq, \glqq Sparse Extended Information Filters \grqq (SEIFs) und \glqq Rao Blackwellized Particle Filters\grqq  (RBPFs). 
 (vgl. \cite{rao} S. 2)

\section{Extended Kalman Filter - SLAM}
Der Kalman Filter ist eine Schätzfunktion für das \glqq linear-quadratic-problem\grqq , welches das Problem der Schätzung des augenblicklichen Zustands eines linearen dynamischen Systems, gestört durch weißes Rauschen, darstellt. Der Kalman Filter wird auch dazu benutzt um die mögliche Zukunft von dynamischen Systemen vorherzusagen, die von Menschen nicht kontrolliert werden können, wie zum Beispiel die Flugbahn von Himmelskörpern, oder der Kurs von gehandelten Rohstoffen. (vgl. \cite{ekf} S.1)

Der Kalman Filter besteht aus drei Schritten. Zuerst wird eine Messung vorhergesagt, welche dann mit der realen Messung verglichen wird. Die resultierende Differenz wird mit der Varianz der Messung gewichtet, um daraus eine neue Schätzung des Zustands zu erhalten. (vgl. \cite{slam_studi} S.13) Der Kalman Filter lässt sich jedoch nur auf lineare Systeme anwenden. Der EKF verwedet für die Vorhersage der Messungen und der Zustände hingegen nichtlineare Funktionen. (vgl. \cite{slam_studi} S.16-17)

Bei Extended Kalman Filter - SLAM ist die Map ein großer Stapel an Vektor und Sensordaten, sowie Zuständen von Landmarks.

\begin{equation}
  x =  \begin{bmatrix}
		R\\
		M
     	\end{bmatrix}
     = \begin{bmatrix}
		R\\
		L_1\\
		...\\
		L_n
     	\end{bmatrix}
\end{equation}

\( R\) ist der Zustand des Roboters und \( M = (L_1, ..., L_n)\)  ist das Set an Zuständen der Landmarks.
Bei EKF wird die Map durch eine gaußsche Variable modelliert, die den Mittelwert und die Kovarianzmatrix des Zustandsvektors verwendet, die jeweils durch \(\overline{x}\) und \(P\) beschrieben werden. Das Ziel ist es die Map \{\(\overline{x}, P\)\} zu allen Zeiten auf dem aktuellsten Stand zu halten.


\begin{equation}
  \overline{x} =  
  		\begin{bmatrix}
		\overline{R}\\
		\overline{M}
     	\end{bmatrix}
     = 
     	\begin{bmatrix}
		\overline{R}\\
		\overline{L_1}\\
		...\\
		\overline{L_n}
     	\end{bmatrix}
     	\quad\quad
     P = 
     	\begin{bmatrix}
		P_{RR} & P_{RM}\\
		P_{MR} & P_{MM}
     	\end{bmatrix}
     = 
     	\begin{bmatrix}
		P_{RR} & P_{RL1} & ... & P_{RLn}\\
		P_{L1R} & P_{L1L1} & ... & P_{L1Ln}\\
		... & ... & ... & ... \\
		P_{LnR} & P_{LnL1} & ... & P_{LnLn}
     	\end{bmatrix}
\end{equation}

Diese Map, die als stochastische Map bezeichnet wird, wird durch die Vorhersage- und Korrekturprozesse des EKF in Stand gehalten. Um eine echte Erkundung der Umgebung zu erreichen, wird der EKF Algorithmus mit einem extra Schritt der Landmark Erkennung und Initialisierung gestartet, bei dem neue Landmarks der Map hinzugefügt werden. Die Landmark Initialisierung erfolgt durch eine Umkehrung der Bewertungsfunktion und der Verwendung dieser und der Ableitungsmatrix, um die beobachteten Landmarks und die benötigten Co- und Crossvarianzen für den Rest der Map zu berechnen. Diese Beziehungen werden dann an den Zustandsvektor und die Kovarianzmatrix angehängt. (vgl. \cite{ekf_slam} S.6-7)

Eine zentrale Einschränkung des EKF basierten SLAM Ansatzes ist die Komplexität der Berechnung. Sensor-Updates benötigen Zeit, quadratisch zur Anzahl der Landmarks \(K\), die zu berechnen sind. Diese Komplexität ergibt sich aus der Tatsache, dass die vom Kalman-Filter verwaltete Kovarianzmatrix \(O(K²)\) Elemente enthält, die alle aktualisiert werden müssen, auch wenn nur ein einzelnes Landmark beobachtet wurde. Diese Komplexität limitiert die Anzahl an Landmarks, die durch diesen Ansatz verarbeitet werden konnen, auf ein paar Hunderte, während natürliche Umgebungsmodelle häufig Millionen von Features enthalten. (vgl. \cite{ekf_problems} S.1)

\url{http://www.iri.upc.edu/people/jsola/JoanSola/objectes/curs_SLAM/SLAM2D/SLAM%20course.pdf}

\section{ORB-SLAM}
TODO

https://arxiv.org/pdf/1502.00956.pdf
https://arxiv.org/pdf/1610.06475.pdf

\section{FAST-SLAM}

Fast-SLAM (Fast SLAM 1.0) zerlegt das SLAM-Problem in ein Lokalisierungsproblem des Roboters und eine Reihe von Landmark Schatzungsproblemen, die auf der Schätzung der Roboterposition beruhen. Fast-SLAM verwendet einen modifizierten Partikelfilter zur Schätzung der \glqq A-posteriori-Wahrscheinlichkeit\grqq  für die Roboterposition. Partikelfilter sind vom Prinzip her ähnlich wie Kalman Filter. Diese werden auch zur Schätzung von Zuständen verwendet, können aber viele verschiedene mögliche Zustände betrachten. Diese Anzahl an Zuständen wird Partikel genannt. Jedes Partikel besitzt wiederum Kalman-Filter, welche die Positionen der Landmarks schätzen, abhängig von der Pfadschätzung. Eine naive Implementation dieser Idee führt zu einem Algorithmus, der \(O(MK)\) Zeit benötigt, wobei \(M\) die Anzahl an Partikeln im Partikel Filter und \( K\) die Anzahl an Landmarks ist. Mit der Verwendung einer Baumstruktur kann die Laufzeit von FastSLAM auf \(O(MlogK)\) reduziert werden, was diesen Algorithmus deutlich schneller als EKF basierte SLAM ALgortihmen macht. (vgl. \cite{ekf_problems} S.1-2, \cite{slam_studi} S.18-19)

Bei Fast-SLAM werden nicht nur die unterschiedlichen geschätzten Positionen verwendet, sondern auch verschiedene Maps der Umgebung betrachtet. Da verschiedenste Maps mit verschieden Warscheinlichkeiten der geschätzten Positionen betrachtet werden, können Fehler in der Map, die sich durch falsche gemessene oder geschätzte Positionen ergeben, durch die Anzahl an verschieden Maps relativieren. Es können sich im Laufe der Zeit Maps, die vorher weniger wahrscheinliche Positionen hatten, als richtig herausstellen. (vgl.  \cite{slam_studi} S.23)

Fast-SLAM 2.0 ist eine weiterentwickelte Variante von Fast-SLAM. Hierbei wird eine Vorauswahl getroffen, berechnet durch einen weiteren Kalman Filtern, in wie weit und mit welcher Varianz die Partikel um den Mittelpunkt verteilt werden. Es ergibt sich eine bessere Auswahl des Mittelpunktes und des Streuradius für den Partikelfilter als in Fast-SLAM 1.0. Diese Methodik reduziert die Anzahl an Partikel und generierten Maps, ist demnach auch schneller berechnet. (vgl. \cite{slam_studi} S.29-30)




\section{SLAM für mobiles Augumented Reality}

Das Ziel von Augumented Reality ist es virtuelle Objekte oder Informationen in die echte Welt zu integrieren, um den Benutzer zusätzliche Informationen in die betrachtete Szene zu liefern. Dazu ist es notwendig, die echte und die virtuelle Welt präzise aneinander auszurichten. Dann kann für jeden Frame aus der Sequenz des Videobildes die genaue Position des mobilen Gerätes bestimmt werden. Um dieses Ziel des exakten Matchings von Realität und generierter virtueller Realität zu erreichen, ist "Camera Localization", also die Lokalisierung der Kamera im dreidimensionalen Raum, anhand von aufgenommenen zweidimensionalen Daten, die Schlüsseltechnologie für alle Augumented Reality Anwendungen. (vgl. \cite{slam_mobile} S.1) Weiterhin ist es wichtig die Umgebung zu kartographieren, um Projektionsflächen für virtuelle dreidimensionale Objekte zu finden. Eine mögliche Lösung für dieses Problem ist die Verwendung von SLAM. In den letzten 20 Jahren hat sich bei SLAM ein Trend gezeigt, bei dem Kameras als einzige exterozeptive Sensoren verwendet werden. Der Hauptgrund dafür ist die Fähigkeit eines kamerabasierten Systems, Tiefeninformationen, Farben, Texturen und Helligkeiten zu erkennen, was Robotern beispielsweise Objekt- oder Gesichtserkennung ermöglicht. Darüber hinaus sind Kameras preiswert, leicht und haben einen geringen Stromverbrauch. (vgl. \cite{survey} S.57) Erst in den letzten Jahren hat sich SLAM in Alltagsanwendungen profilieren können, hauptsächlich wegen dem Aufkommen von Smartphones. Smartphones sind mobil und haben die nötige Rechenleistung um SLAM Aufgaben in Echtzeit auszuführen. Dies hat zu einer Erweiterung der Forschungsmöglichkeiten von SLAM geführt und hat es zu einer robusteren Technologie gemacht. Weiterhin verfügen Smartphones über eine ganze Reihe an Sensoren, wie Beschleunigungssensoren, Gyroskop, Magentometer oder GPS, mit denen die visuellen Daten ergänzt werden können, um die Map genauer und weniger anfällig für innere Drifteffekte zu machen, was jedoch wieder eigene komplexe Probleme bei der Verbindung von verschiedenen Sensordaten mit sich bringt. (vgl. \cite{ar_slam} S.4) 

Die Verwendung von SLAM im mobilen Bereich hat jedoch einige Probleme zu meistern. Durch die Ungenauigkeiten der absoluten Lokalisierung, ist es schwierig kontextsensitive Augmentation zu erzeugen. Auch wenn es möglich ist bekannte Objekte im Raum während des Trackings zu bestimmen, und die Information um diese herum zu zeigen, ist es fast unmöglich den gesamten Raum einer Szenerie zu bestimmen. Das zweite Problem liegt in der Ergonomie und Benutzbarkeit der Anwendung durch den Endbenutzer. Typischerweise soll ein Endverbraucher keinem komplexem Protokoll folgen müssen, um sich selbst in der Szene zu lokalisieren. Das System benötigt also eine Verfahren zur schnellen und robusten Lokalisierung im Raum. (vgl. \cite{slam_mobile} S.1)



\url{https://hal.inria.fr/hal-00994756/document}
\url{https://pdfs.semanticscholar.org/00f4/41387f04f40aad6491ce23bdeb0ece17d12e.pdf}


file:///C:/Users/Daniel/Downloads/AIREVSLAMSurvey.pdf


\section{SLAM als Core für viele AR APIs}

file:///C:/Users/Daniel/Downloads/COMPARATIVESTUDYOFAUGMENTEDREALITY.pdf

  \chapter{Vergleich von Photogrammetrie und SLAM}

Es ist inzwischen allgemein bekannt, dass Photogrammetrie und geometrische Computer Vision, in dessen Bereich SLAM einzuordnen ist, zwei eng zusammenhängende Disziplinen sind. Sie haben viele ähnliche Aufgabenstellungen und Ziele, wie Kalibrierung, Orientierung und Rekonstruktion. Viele Arbeiten und Forschungen beziehen sich auf beide Gebiete, wie relative Orientierung (Philip, 1996; Nistèr, 2004), die räumliche Analyse von Einzelbildern (Masry, 1981; Lepetit et al., 2009), Feature Erkennung  (Förstner \& Gülch, 1986; Lowe, 2004) oder etwa der Bündelblockausgleich  (Triggs et al., 2000). Dabei sollte beachtet werden, dass viele dieser Probleme erst in der Photogrammetrie untersucht und beschrieben worden sind und erst später in der Computer Vision signifikant weiter entwickelt wurden. Dies hat die Kommunikation zwischen beiden Fachbereichen gefördert (vgl. \cite{ph_vs_cv} S.93).


\section{Ähnlichkeiten und Unterschiede zwischen Visual SLAM, Photogrammetrie und SfM}

In den Kapiteln 3 und 4 wurde ein Überblick über die Funktionsweise von Photogrammetrie und Computer Vision, mit Fokus auf SLAM, gegeben. Um einen Gesamtüberblick über diese Themenbereiche geben zu können, sowie die Abgrenzung der Bereiche zu ermöglichen, werden nun die Ähnlichkeiten und Unterschiede der Technologien beschrieben. Der Zusammenhang der Inhalte von Photogrammetrie und SLAM, bezieht sich hauptsächlich auf die Theorie und Anwendung der Zentralprojektion. Die Gemeinsamkeiten der beiden Felder sind Kamera Kalibrierung, Positionsbestimmung der Kamera, Feature Erkennung und Matching, sowie Modellerstellung der Umwelt. Doch warum ist SLAM ein Teil von Computer Vision und nicht nur ein Teilbereich der Photogrammetrie?

Obwohl bemerkenswerte Entwicklungen in beiden Bereichen gemacht wurden, ist das Verhältnis zwischen den beiden Diszipllinen noch sehr distinkt. Die Unterscheidung kann auf die unterschiedlichen Traditionen, Philosophien und Anwendungsbereiche zurückgeführt werden. In der Photogrammetrie, die ihren Ursprung in der Vermessung und Kartierung hat, ist Genauigkeit und Präzision das Hauptaugenmerk und die meisten photogrammetrischen Arbeiten sind im Bereich des Post-Processing anzusiedeln. Für Computer Vision hingegen ist die Genauigkeit eher zweitrangig und wird meist nur durch die Anwendung definiert. Die Verarbeitungsgeschwindigkeit stellt in diesem Bereich jedoch den kritischen Faktor dar, da viele Anwendungen in Echtzeit ablaufen müssen, wie beispielsweise Objekterkennung, Roboternavigation oder Positionsbestimmung im Raum. Auch wenn photogrammetrische Verfahren immer mehr in Echtzeitanwendungen beteiligt sind, sind die beiden Ansätze grundverschieden. Weiterhin ist die Geometrie nach wie vor ein Hauptanliegen der Photogrammetrie, während in der Computer Vision maschinelles Lernen und Erkennung im Vordergrund steht. Aus mathematischer Sicht sind Photogrammetrie und Computer Vision zwei verschiedene Repräsentationen und Lösungen der Kamerageometrie (vgl. \cite{ph_vs_cv} S.93-94).

Doch wie kann man die Gebiete nun voneinander abgrenzen? Alle diese Verfahren beschreiben eigentlich unterschiedliche Ansätze für das gleiche Problem: die gleichzeitige Lokalisierung eines Sensors in Bezug auf seine Umgebung bei gleichzeitiger Erstellung einer 3D-Karte dieser Umgebung. Das einzigartige Merkmal von monokularem \textbf{visuellem SLAM} ist nicht, dass die Kameraposition und Szenenstruktur wiederhergestellt werden, sondern dass der Prozess simultan, rekursiv und in Echtzeit durchgeführt wird. Dies impliziert dynamisches Abtasten, wobei die Karte so aufgebaut sein muss, dass sie im laufenden Prozess die Positionsbestimmung von neu aufgenommenen Bildern integrieren kann, sowie die Triangulation neuer Punkte unterstützt. SLAM Konzepte sind deswegen aus den Bereichen wie der Selbsterkundung von Robotern, der automatischen Fahrzeugnavigation oder im nicht fotografischen geometrischen Kontext, wie etwa dem Handlaser Scanning entstanden und weiterentwickelt worden. Wenn der Echtzeitaspekt von monokularem visuellem SLAM, was auch als visuelle Odometrie bezeichnet werden kann, wegfällt, bleiben SfM und Photogrammerie übrig. \textbf{Structure from Motion} wird als eine der herausragendsten Leistungen der Computer Vision bezeichnet. Es muss jedoch bedacht werden, dass die Entwicklung dieser Feature basierten Matching- und Lokaliserungstechnik von 3D Punkten im Raum wenig mit der Bildbasierten 3D-Messung, das heißt der Photogrammetrie zu tun hatte, sondern mehr mit der räumlichen Archivierung von ungeordneten Sammlungen von Bildern, oft von unbekannten Kameras. \\ Bei SfM wird hauptsächlich die Kameralokalisierung und nicht die Erzeugung von Punktwolken betont und es wurde die Notwendigkeit der Kamerakalibrierung erkannt. Die technischen Prinzipien, welche die Grundlagen der \textbf{Photogrammetrie} bilden, sind bei der Entwicklung von Structure from Motion nicht berücksichtigt. Dazu zählen zum Beispiel geometrisches Netzwerkdesign, metrische Kamerabetrachtung, Genauigkeitsoptimierung und Varianzausbreitung, systematische Fehlerkompensation und grobe Fehlererkennung, ortsunabhängige Kamerakalibrierung und die Einführung von Beobachtungsredundanz zur Erhöhung der Zuverlässigkeit und Qualitätskontrollverfahren in allen Phasen der photogrammetrischen Datenverarbeitungspipeline, die alle durch strenge, unveränderlich nichtlineare mathematische Modelle gestützt werden (vgl. \cite{vergleich_fraser} S.1-2).

C. Fraser beschreibt den Unterschied zwischen Structure from Motion und Photogrammetrie folgendermaßen \cite{vergleich_fraser}: \\ \\ \textit{ \glqq It could be well argued that a data processing pipeline that is initiated with SfM-based camera localisation, but then follows the principles of photogrammetry, is no-longer SfM per se, but indeed standard photogrammetry. To call a photogrammetry measurement an SfM solution not only implies that the solution is less ‘metric’ (i.e. accurate and reliable) than may be the case, but also attributes more to the SfM process than what is actually there.\grqq }

Structure from Motion ist also ein leistungsstarker Ansatz zur Lösung des Lokalisationsproblems, ausgehend von einer unbekannten Umgebung. Die neu gewonnenen Erkenntnisse hatten dann in der Photogrammetrie wenig Einfluss, da diese schon seit langem ihre eigenen präzisen Algorithmen zur Sensorausrichtung hat. Die Bezeichnungen SLAM, SfM und Photogrammetrie haben also großteils unterschiedliche Bedeutungen, auch wenn es große konzeptionelle und algorithmische Überschneidungen gibt. Weder SLAM noch SfM umfassen den gesamten Prozess der metrischen bildbasierten 3D-Messung, Objektrekonstruktion und Kartierung, welche heute die automatisierte Photogrammetrie auszeichnet (vgl. \cite{vergleich_fraser} S.2).



  \chapter{Implementation einer AR Anwendung für Android}

Im Rahmen dieser Arbeit ist eine Applikation für Android entstanden.

\section{Verwendete Hard und Software}

\subsection{Ar Core}

\subsection{Sceneform}

\subsection{Google Location Service}

\subsection{Dexter}

\subsection{Volley}

\section{Implentierung}




  \chapter{Praxistest}


\section{Anwendungsbeispiel}

\section{Benchmarks}
  \include{kapitel6}
  \chapter{Zusammenfassung}


Fortschritte in der Photogrammetrie sind viel zu sehr mit den Fortschritten der Computer Vision verflochten, als dass sich die Konvergenz der beiden Disziplinen umkehren wird. Photogrammetrie und Computer Vision haben die Extraktion und Rekonstruktion von Daten aus Bildmaterial zu unterschiedlichen Zeiten und mit unterschiedlichen Zielen begonnen. Als sich dann 3D-Modelle als Referenzziel herausstellten, wurde der Austausch von Ansätzen und Techniken zwischen den Disziplinen vorangetrieben (vgl. \cite{state_of_art} S.9). Dies hat dazu geführt, dass Photogrammetrie und Computer Vision verfahrenstechnisch kaum mehr unterscheidbar sind, auch wenn die ursprünglichen Anwendungsgebiete verschieden sind. 

Im Rahmen dieser Arbeit wurden Algorithmen und Konzepte aus der Photogrammetrie und der Computer Vision beschrieben und verglichen. Anschließend wurde ein Überblick über die Ziele, Unterschiede und Gemeinsamkeiten von Photogrammetrie, SLAM und SfM gegeben. Es hat sich gezeigt, dass sich diese Verfahren inhaltlich stark überschneiden, auch wenn die Fachrichtung, der historische Kontext und die Ziele von Photogrammetrie und SLAM grundverschieden sind. 

Im Praxisteil wurde eine Applikation erstellt, welche ARCore verwendet. ARCore implementiert SLAM und verwendet damit auch photogrammetrische Verfahren. Photogrammetrie kann also in Augmented Reality Anwendungen verwendet werden, auch wenn die Bezeichnung irreführend ist. SLAM könnte man als Echtzeit-Photogrammetrie bezeichnen, welche nicht auf größtmögliche Genauigkeit, sondern auf Schnelligkeit und Stabilität ausgelegt ist. In Zukunft wird sich die Konvergenz der in dieser Arbeit beschrieben Disziplinen noch verstärken, die Kategorisierung in die verschiedenen Fachbereiche ist jedoch mit Sicht auf die Anwendungsgebiete, Philosophien und den historischen Ursprung, sinnvoll.

Vielen Dank an Prof. Dr. Tobias Lenz und Prof. Dr. Klaus Jung für die Betreuung und Korrektur dieser Arbeit.
	

  % ... weitere Kapitel
 
  % Literaturverzeichnis
  \phantomsection
  \addcontentsline{toc}{chapter}{Literaturverzeichnis}
  \begin{thebibliography}{10}
  
  
    \bibitem[1]{photo} Heipke, C. (2017), \glqq Photogrammetrie und Fernerkundung \grqq, 1.Auflage, Berlin: Springer Verlag, S. 5-7.
    
    \bibitem[2]{slam} Hugh Durrant-Whyte, Tim Bailey (2006), Simultaneous Localisation and Mapping (SLAM): Part I The Essential Algorithms. URL: \url{https://people.eecs.berkeley.edu/~pabbeel/cs287-fa09/readings/Durrant-Whyte_Bailey_SLAM-tutorial-I.pdf} (Zuletzt abgerufen am 09.07.2019)
    

	\bibitem[3]{slam_mobile} P. Martin, E. Marchand, P. Houlier, I. Marchal. Mapping and re-localization for mobile augmented reality. IEEE Int. Conf. on Image Processing, Oct 2014, Paris, France.
URL: \url{https://hal.inria.fr/hal-00994756/document} (Zuletzt abgerufen am 09.07.2019) 

	\bibitem[4]{ekf_slam} Joan Sol`a, (2014), Simulataneous localization and mapping with the extended Kalman filter. URL: \url{http://www.iri.upc.edu/people/jsola/JoanSola/objectes/curs_SLAM/SLAM2D/SLAM\%20course.pdf} (Zuletzt aufgerufen am 10.07.2019)
    
    \bibitem[5]{ekf} Mohinder S. Grewal, Angus P. Andrews (2001), Kalman Filtering: Theory and Practice with MATLAB. URL: \url{http://staff.ulsu.ru/semoushin/_index/_pilocus/_gist/docs/mycourseware/13-stochmod/2-reading/grewal.pdf} (Zuletzt aufgerufen am 10.07.2019)

	\bibitem[6]{ekf_problems} M. Montemerlo, S. Thrun, D. Koller, B. Wegbreit (2002). FastSLAM: A Factored Solution to the Simultaneous Localization and Mapping Problem. URL: \url{http://robots.stanford.edu/papers/montemerlo.fastslam-tr.pdf} (Zuletzt aufgerufen am 11.07.2019) 
	
	\bibitem[7]{rao} G. Grisetti, G.D. Tipaldi, C. Stachniss, W. Burgard, D. Nardi (2007). Fast and accurate SLAM with Rao-Blackwellized particle filters. URL: \url{http://srl.informatik.uni-freiburg.de/publicationsdir/grisettiRAS07.pdf} (Zuletzt aufgerufen am 11.07.2019)

\bibitem[8]{slam_studi} M. Mengelkoch (2007). Implementieren des FastSLAM Algorithmus zur
Kartenerstellung in Echtzeit. URL: \url{https://kola.opus.hbz-nrw.de/opus45-kola/frontdoor/deliver/index/docId/183/file/sa-00.pdf} (Zuletzt aufgerufen am 11.07.2019)

\bibitem[9]{survey} J. Fuentes-Pacheco, J. Ruiz-Ascencio, J.M. Rendón-Mancha (2012). Visual simultaneous localization and mapping: a survey. In: J.M. Artif Intell Rev (2015) 43: 55. Springer Netherlands. DOI: \url{https://doi.org/10.1007/s10462-012-9365-8}

\bibitem[10]{ar_slam} J. Halvarsson, (2018), Using SLAM-based technology to improve directional navigation in an Augmented Reality game. URL: \url{http://umu.diva-portal.org/smash/get/diva2:1245293/FULLTEXT01.pdf} (Zuletzt aufgerufen am 11.07.2019)

\bibitem[11]{ar_vr} P. Milgram, H. Takemura, A. Utsumi, F. Kishino (1994), Augmented Reality: A class of displays on the reality-virtuality continuum. URL: \url{http://etclab.mie.utoronto.ca/publication/1994/Milgram_Takemura_SPIE1994.pdf} (Zuletzt aufgerufen am 16.07.2019)

\bibitem[12]{sdks} A. Hanafi, L. Elaachak, M. Bouhorma (2019), A comparative Study of Augmented Reality SDKs to Develop an Educational Application in Chemical Field. DOI: 10.1145/3320326.3320386

\bibitem[13]{comparative_sdks} D. Amin, S. Govilkar (2015), Comparative Study of Augmented Reality
SDK’s. International Journal on Computational Sciences \& Applications (IJCSA) Vol.5, No.1, February 2015 URL: \url{https://pdfs.semanticscholar.org/e752/17e8897cb46b466d6ba83e909cca4ecff8f2.pdf} (Zuletzt aufgerufen am 16.07.2019)

\bibitem[14]{model_based} M. Lowney, A. S. Raj (2016), Model Based Tracking for Augmented Reality
on Mobile Devices. URL: \url{https://web.stanford.edu/class/ee368/Project_Autumn_1617/Reports/report_lowney_raj.pdf} (Zuletzt aufgerufen am 16.07.2019)

\bibitem[15]{natural_feature} S. Ćuković, M. Gattullo, F. Pankratz, G. Devedzic, E. Carrabba, K. Baizid (2015), Marker Based vs. Natural Feature Tracking Augmented Reality Visualization of the 3D Foot Phantom. URL: \url{https://www.researchgate.net/publication/278668320_Marker_Based_vs_Natural_Feature_Tracking_Augmented_Reality_Visualization_of_the_3D_Foot_Phantom} (Zuletzt aufgerufen am 17.07.2019)

\bibitem[16]{homography} Basic concepts of the homography explained with code. URL: \url{https://docs.opencv.org/3.4.1/d9/dab/tutorial_homography.html} (Zuletzt aufgerufen am 17.07.2019)

\bibitem[17]{camera_pose} M. Maidi, J.Y. Didier, F. Ababsa, M. Mallem (2008), A performance study for camera pose estimation using visual marker based tracking. URL: \url{https://www.academia.edu/13152554/A_performance_study_for_camera_pose_estimation_using_visual_marker_based_tracking}
(Zuletzt aufgerufen am 18.07.2019)

\bibitem[18]{vorraussetzungen} A. Pinz, M. Brandner, H. Ganster, A. Kusej, P. Lang, M. Ribo (2002) Hybrid Tracking for Augmented Reality. URL: \url{https://www.researchgate.net/publication/229025765_Hybrid_tracking_for_augmented_reality} (Zuletzt aufgerufen am 18.07.2019)

\bibitem[19]{fast} E. Rosten, T. Drummond (2006) Machine learning for high-speed corner detection. URL: \url{https://www.edwardrosten.com/work/rosten_2006_machine.pdf} (Zuletzt aufgerufen am 18.07.2019)

\bibitem[20]{fiundations_pg} K. Schindler (2014) Mathematical Foundations of
Photogrammetry. URL: \url{https://ethz.ch/content/dam/ethz/special-interest/baug/igp/photogrammetry-remote-sensing-dam/documents/pdf/math-of-photogrammetry.pdf} (Zuletzt aufgerufen am 19.07.2019)

\bibitem[21]{pp} A.S. Alturki, J.S. Loomias (2016) Camera Principal Point Estimation from Vanishing Points. DOI: 10.1109/NAECON.2016.7856820 (Zuletzt aufgerufen am 19.07.2019)

\bibitem[22]{exterior_review} P. Grussenmeyer, O. Al Khalil (2002) Solutions for Exterior Orientation in Photogrammetry: A Review. Photogrammetric Record, 17(100):615-634. URL: \url{https://hal.archives-ouvertes.fr/hal-00276983/document} (Zuletzt aufgerufen am 19.07.2019)

\bibitem[23]{comparative_conditions_study} K.L.A. El-Ashmawy (2015). A comparison study between collinearity condition, coplanarity condition, and direct linear transformation (DLT) method for camera exterior orientation parameters determination. Geodesy and Cartography, 41(2), 66–73. URL:\url{https://journals.vgtu.lt/index.php/GAC/article/view/2837/2334} (Zuletzt aufgerufen am 19.07.2019)

\bibitem[24]{coll_exterior} E.E. Elnima (2015) A solution for exterior and relative orientation
in photogrammetry, a genetic evolution approach. Journal of King Saud University – Engineering Sciences. URL: \url{https://core.ac.uk/download/pdf/82822280.pdf}  (Zuletzt aufgerufen am 22.07.2019)

\bibitem[25]{lev_efficient} M. Lourakis, A. Argyros (2005) Is Levenberg-Marquardt the Most Efficient Optimization Algorithm for Implementing Bundle Adjustment? URL: \url{https://www.ics.forth.gr/_publications/0201-P0401-lourakis-levenberg.pdf} (Zuletzt aufgerufen am 20.08.2019) 

\bibitem[26]{state_of_art} G. Forlani, R. Roncella, C. Nardinocchi (2015) Where is photogrammetry heading to? State of the art and trends. Rendiconti Lincei, 26(S1), 85–96. DOI:10.1007/s12210-015-0381-x  (Zuletzt aufgerufen am 24.07.2019)

\bibitem[27]{dlt} K. L. El-Ashmawy (2018) Using direct linear Transformation (DLT) Method for
aerial Photogrammetry Applications. Geodesy and Cartography 2018 Volume 44 Issue 3: 71–79. URL: \url{https://journals.vgtu.lt/index.php/GAC/article/download/1629/5048} (Zuletzt aufgerufen am 24.07.2019)

\bibitem[28]{nonlinear_1} L. Vandenberge (2018) 13. Nonlinear least squares. URL: \url{http://www.seas.ucla.edu/~vandenbe/133A/lectures/nlls.pdf} (Zuletzt aufgerufen am 20.08.2019)

\bibitem[29]{least_quares} A. W. Gruen (1985) Adaptive least Squares Correlations:
A powerful Image Matching Technique. URL: \url{https://www.researchgate.net/publication/265292615_Adaptive_Least_Squares_Correlation_A_powerful_image_matching_technique}  (Zuletzt aufgerufen am 24.07.2019)

\bibitem[30]{nonlinear_2} S. Boyd (2016) Nonlinear Least Squares. URL: \url{https://stanford.edu/class/ee103/lectures/nlls_slides.pdf} (Zuletzt aufgerufen am 24.07.2019)

\bibitem[31]{approx_gn} S. Gratton, A. S. Lawless, N. K. Nichols (2007) Approximate Gauss–Newton Methods for Nonlinear Least Squares Problems. URL: \url{https://www.researchgate.net/publication/220133629_Approximate_Gauss-Newton_Methods_for_Nonlinear_Least_Squares_Problems} (Zuletzt aufgerufen am 30.07.2019)

\bibitem[32]{lev_mar} H. P. Gavin (2019) The Levenberg-Marquardt algorithm for
nonlinear least squares curve-fitting problems. URL: \url{http://people.duke.edu/~hpgavin/ce281/lm.pdf} (Zuletzt aufgerufen am 30.07.2019)

\bibitem[33]{levenberg} K. Levenberg (1944) A method for the solution of certain non-linear problems in least squares. URL: \url{https://www.ams.org/journals/qam/1944-02-02/S0033-569X-1944-10666-0/} (Zuletzt aufgerufen am 30.07.2019)

\bibitem[34]{bundle_adjustment} N. Börlin, P. Grussenmeyer (2013) Bundle Adjustment With and Without Damping. The Photogrammetric Record, 28(144), 396–415. DOI: 10.1111/phor.12037 (Zuletzt aufgerufen am 07.08.2019)

\bibitem[35]{robust_feature} A. Abbas, S. Ghuffar (2018) Robust Feature Matching in terrestrial Image Sequences. The International Archives of the Photogrammetry, Remote Sensing and Spatial Information Sciences, Volume XLII-3, URL: \url{https://www.researchgate.net/publication/324855190_ROBUST_FEATURE_MATCHING_IN_TERRESTRIAL_IMAGE_SEQUENCES} (Zuletzt aufgerufen am 07.08.2019)

\bibitem[36]{det_des} F. Remondino (2006) Detectors and descriptors for photogrammetric applications. URL: \url{http://3dom.fbk.eu/sites/3dom.fbk.eu/files/pdf/remondino_ISPRS_III_06.pdf} (Zuletzt aufgerufen am 07.08.2019)

\bibitem[37]{old_new_feature} A. Lingua, D. Marenchio, F. Nex (2009) A comparison between “old and new” feature extraction and matching techniques in Photogrammetry. URL: \url{https://pdfs.semanticscholar.org/b9c3/067b6fc111e1cc02f42357d5fb459a642152.pdf} (Zuletzt aufgerufen am 07.08.2019)

\bibitem[38]{efficient_bundle} Z. Qu (2018) Efficient Optimization for
Robust Bundle Adjustment. URL: \url{https://vision.in.tum.de/_media/members/demmeln/qu2018msc.pdf}  (Zuletzt aufgerufen am 08.08.2019)

\bibitem[39]{loop_closure} A. Bokovoy, K. Yakovlev (2017) Original Loop-Closure Detection Algorithm for Monocular vSLAM. Analysis of Images, Social Networks and Texts, 210–220. URL: \url{https://arxiv.org/pdf/1707.04771.pdf} (Zuletzt aufgerufen am 08.08.2019)

\bibitem[40]{orb_slam} M. Andersson, M. Baerveldt (2018) Simultaneous localization and mapping
for cars using ORB2-SLAM. URL: \url{http://publications.lib.chalmers.se/records/fulltext/256291/256291.pdf} (Zuletzt aufgerufen am 08.08.2019)

\bibitem[41]{orbslam_og}  R. Mur-Artal, J. M. M, Montiel (2015) ORB-SLAM: a Versatile and Accurate Monocular SLAM System. URL: \url{https://arxiv.org/pdf/1502.00956.pdf}(Zuletzt aufgerufen am 08.08.2019)

\bibitem[42]{brief} M. Calonder, V. Lepetit, C. Strecha, P. Fua (2010) BRIEF: Binary Robust Independent Elementary Features. URL: \url{https://www.researchgate.net/publication/221304115_BRIEF_Binary_Robust_Independent_Elementary_Features} (Zuletzt aufgerufen am 08.08.2019)

\bibitem[43]{ph_vs_cv} R. Tang (2013) Mathematical Methods for Camera Self-Calibration
in Photogrammetry and Computer Vision. URL: \url{https://elib.uni-stuttgart.de/bitstream/11682/3934/1/Tang_Uni.pdf} (Zuletzt aufgerufen am 12.08.2019)

\bibitem[44]{sfm} S. N. Sinha1, D. Steedly, R. Szeliski1 (2016) A multi-stage linear approach to structure from motion. URL: \url{https://www.microsoft.com/en-us/research/wp-content/uploads/2016/07/sinhaRMLE10_linearSfm.pdf} (Zuletzt aufgerufen am 13.08.2019)

\bibitem[45]{sfm_photo} M. J. Westobya, J. Brasington, N. F. Glasser, M. J. Hambrey, J. M. Reynolds (2012) 'Structure-from-Motion' photogrammetry: A low-cost, effective tool for geoscience applications. URL: \url{https://www.sciencedirect.com/science/article/pii/S0169555X12004217} (Zuletzt aufgerufen am 13.08.2019)

\bibitem[46]{vergleich_fraser} C. Fraser (2018) SLAM, SFM AND PHOTOGRAMMETRY: WHAT’S IN A NAME? URL: \url{https://www.isprs.org/tc2-symposium2018/images/ISPRS-Interview-Fraser.pdf} (Zuletzt aufgerufen am 13.08.2019)

\bibitem[47]{pose} F. Herranz, K. Muthukrishnan, K. Langendoen (2011) Camera pose estimation using particle filters. URL: \url{https://www.researchgate.net/publication/229033775_Camera_pose_estimation_using_particle_filters} (Zuletzt aufgerufen am 19.08.2019)

\bibitem[48]{markers} H. Le, M. Nguyen, H. Tran, W. Yeap (2017) Pictorial AR Tag with Hidden Multi-Level Bar-Code and Its Potential Applications. URL: \url{https://www.mdpi.com/2414-4088/1/3/20} (Zuletzt aufgerufen am 19.08.2019)

\bibitem[49]{sift} D. G. Lowe (1999) Object Recognition from Local Scale-Invariant Features. URL: \url{https://www.cs.ubc.ca/~lowe/papers/iccv99.pdf} (Zuletzt aufgerufen am 19.08.2019)

\bibitem[50]{pose_estimation} E. Marchand, H. Uchiyama, F. Spindler (2016) Pose Estimation for Augmented Reality: A Hands-On Survey. IEEE Transactions on Visualization and Computer Graphics, Institute of Electrical and Electronics Engineers, 2016, 22 (12), pp.2633 - 2651. URL: \url{https://hal.inria.fr/hal-01246370/document} (Zuletzt aufgerufen am 19.08.2019)






	\end{thebibliography}  
  \newpage
  
  % Anhang
  \phantomsection
  \addcontentsline{toc}{chapter}{Abbildungsverzeichnis}
  \listoffigures
  \newpage
 
  
  %\include{anhang}
\end{document}    